%!TEX root = ../kursovaya.tex \phantomsection
\section*{Введение}  \addcontentsline{toc}{section}{Введение}
 	Данная работа посвящена исследованию расстояния Громова-Хаусдорфа, 
определенному для облаков в классе Громова-Хаусдорфа. Расстояние Громова-Хаусдорфа традиционно рассматривается на пространстве компактных метрических пространств с точностью до изометрии. Ограниченное на это пространство, данное расстояние становится метрикой. Для того, чтобы перенести это свойство на все метрические пространства, необходимы две конструкции. 
Рассматривается собственный класс всех метрических пространств с точностью до нулевого
расстояния между пространствами, обозначаемый $\mathcal{GH}_0$. Это позволяет превратить расстояние Громова-Хаусдорфа в псевдометрику. Затем, для того, чтобы избежать бесконечных расстояний, собственный класс $\mathcal{GH}_0$ разбивают на классы эквивалентности, называемые облаками. Отношением эквивалентности между метрическими пространствами в данном случае служит конечность расстояния между ними. Тот факт, что это именно отношение эквивалентности, следует из определения псевдометрики. 
Итак, внутри каждого облака расстояние Громова-Хаусдорфа становится метрикой. 

	В работе показано, что облака не является, при этом, метрическими пространствами, так как
являются собственными классами, а не множествами. Несмотря на это, конструкция расстояния 
Громова-Хаусдорфа переносится на них успешно. 
 	
	В данной работе были получены некоторые свойства этого расстояния, 
в частности, основное внимание уделено связи между расстоянием между 
облаками и их стационарными группами. Особую роль здесь играет облако
ограниченных метрических пространств, для которого можно посчитать расстояние
до некоторых облаков специального вида. В данной работе представлен 
подсчет расстояния до облака, содержащего действительную прямую.

