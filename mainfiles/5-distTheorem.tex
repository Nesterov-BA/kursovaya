%!TEX root = ../kursovaya.tex

\section{Подсчет расстояния между облаками $[\Delta_1]$ и $[\mathbb{R}]$}
\begin{theorem} Расстояние между облаками $[\Delta_1]$ и $[\mathbb{R}]$ равно
бесконечности.
\end{theorem} 
\begin{proof} У облаков $[\Delta_1]$ и $[\mathbb{R}]$ стационарные группы имеют
нетривиальное пересечение, и, по следствию 1, расстояние между ними может быть
равно либо $0$, либо $\infty$. \\ Для доказательства утверждения теоремы
достаточно будет показать, что расстояние между ними не равно $0$.  Для этого
необходимо установить, что между ними не может существовать соответствия со
сколь угодно малым искажением. Итак, пусть $R$ --- соответствие между
$[\Delta_1]$ и $[\mathbb{R}]$, $\dis R = \epsilon < \infty$.
Зафиксируем $Y$ из $R(\Delta_1)$. По теореме 2.1 расстояние между $Y$ и $\mathbb{R}$ не больше $2\epsilon$.\\  По теореме 3.1 пространства $\mathbb{Z}$, $\widetilde{\mathbb{R}}$ лежат от $\mathbb{R}$ на расстоянии не большем, чем $\frac{1}{2}$. При этом,
расстояние между этими пространствами строго больше $\frac{1}{2}$. Обозначим за
$r$ максимум из расстояний от этих пространств до $\mathbb{R}$:
	$$r = \max\big\{|\mathbb{Z}, \mathbb{R}|,|\widetilde{\mathbb{R}}, \mathbb{R}|\big\}\le \frac{1}{2} < |\mathbb{Z}, \widetilde{\mathbb{R}}|.$$
	Неравенство означает, что существует $c > 0$ такое, что
$|\mathbb{Z}, \widetilde{\mathbb{R}}| = (1 + c)r.$ \\ Вместе с $\mathbb{Z}$ и
$ \widetilde{\mathbb{R}}$ рассмотрим их прообразы $X_1 \in R^{-1}(\mathbb{Z})$,
$ X_2 \in R^{-1}(\widetilde{\mathbb{R}})$.  \\ Получаем следующую цепочку
неравенств:
	$$|X_1, \Delta_1| \le |\mathbb{Z}, Y| + \epsilon \le |\mathbb{Z}, \mathbb{R}| + |\mathbb{R}, Y| +\epsilon \le r + 2\epsilon + \epsilon = r + 3\epsilon.$$
	Аналогичное неравенство имеет место для $X_2$, при этом
	$$|X_1, X_2|  \ge |\mathbb{Z}, \widetilde{\mathbb{R}}| - \epsilon = (1+c)r - \epsilon.$$
	По замечанию 1.4:
	$$|X_1, X_2| \le \max\big\{ |X_1, \Delta_1|, |X_2, \Delta_1| \big\},$$
	$$\Updownarrow$$
	$$(1+c)r - \epsilon\le r + 3\epsilon,$$
	$$\Updownarrow$$
	$$\epsilon \ge \frac{cr}{4}.$$
	Мы получаем оценку снизу для $\epsilon = \dis R$. Это означает, что
искажение не может быть произвольно малым, и следовательно расстояние между
пространствами не может быть равно 0. Значит, оно равно бесконечности.
	
\end{proof}
