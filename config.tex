%!TEX root = kursovaya.tex
% Добавьте ссылку на файлы с текстом работы
% Можно использовать команды:
%   \input или \include
% Пример:
%    \input{mainfiles/1-section} или \include{mainfiles/2-section}
% Команда \input позволяет включить текст файла без дополнительной обработки
% Команда \include при включении файла добавляет до него и после него команду
% перехода на новую страницу. Кроме того, она позволяет компилировать каждый файл
% в отдельности, что ускоряет сборку проекта.
% ВАЖНО: команда \include не поддерживает включение файлов, в которых уже содержится команда \include,
% т.е. не возможен рекурсивный вызов \include
\newcommand*{\Source}{
    %!TEX root = ../kursovaya.tex
\phantomsection
\section*{Введение} 
\addcontentsline{toc}{section}{Введение}
	Понятие облаков метрических пространств является частью теории расстояния Громова--Хаусдорфа. Так как облака сами по себе являются метрическими классами, естественно ввести на собственном классе облаков обобщенную псевдометрику, являющуюся аналогом расстояния Громова-Хаусдорфа.
	В данной работе приведены некоторые результаты изучения ее свойств, а также поднимается вопрос об изометрическом изоморфизме различных облаков, тесно связанный с вопросом расстояния между ними. 
      %!TEX root = ../kursovaya.tex

\section{Основные определения и предварительные результаты} Пусть $X$ и $Y$ ---
метрические пространства. Тогда между ними можно задать расстояние, называемое
расстоянием Громова--Хаусдорфа. Введем два его эквивалентных (\cite{Lectures})
определения.  \Def{Пусть $X$, $Y$ --- метрические пространства с метриками
$\rho_X$ и $\rho_Y$. \emph{Соответствием} $R$ между этими пространствами
называется подмножество декартового произведения $X\times Y$ такое, что
проекторы $\pi_X \colon (x,y) \mapsto x$, $\pi_Y\colon (x,y) \mapsto y$ являются
сюрьективными. Множество всех соответствий между $X$ и $Y$ обозначается
$\mathcal{R}(X,Y).$} \Def{ Пусть $R$ --- соответствие между $X$ и $Y$.
\emph{Искажением} соответствия $R$, $\dis{R}$ является
величина $$ \dis{R} = \sup{\Bigl\{ \big| |xx'| - |yy'| \big| : (x, y), (x', y') \in R\Bigr\}}.$$
Тогда \emph{расстояние Громова--Хаусдорфа} $d_{GH}(X,Y)$ можно определить
следующим
образом $$ d_{GH}(X,Y) = \frac{1}{2}\inf \bigl\{\dis{R} : R \in \mathcal{R}(X,Y)\bigr\}.$$}
	
	\Def{\emph{Реализацией} пары метрических пространств $(X,Y)$ назовем тройку
метрических пространств $(X',Y',Z)$ таких, что $X' \subset Z$, $Y' \subset Z$,
$X'$ изометрично $X$, $Y'$ изометрично $Y$. \emph{Расстоянием Громова-Хаусдорфа}
$d_{GH}(X,Y)$ между метрическими пространствами $X, Y$ является точная нижняя
грань чисел $r$ таких, что существует реализация $(X',Y',Z)$ и \\$d_H(X', Y')
\le r$, где $d_H$ --- расстояние Хаусдорфа.}

	Рассмотрим собственный класс всех метрических пространств и отождествим в нем между собой все метрические пространства, находящиеся на нулевом расстоянии друг от друга. Обозначим получившийся класс $\mathcal{GH}_0$. На нем расстояние Громова -- Хаусдорфа будет являться обобщенной метрикой.
	
	\Def{(\cite{TuzhBog1}) В классе $GH$ рассмотрим следующее отношение:
$X \thicksim Y \Leftrightarrow d_{GH}(X, Y) < \infty$. Нетрудно убедиться, что
оно будет отношением эквивалентности. Классы этой эквивалентности называются
\emph{облаками}. Облако, в котором лежит метрическое пространство $X$ будем
обозначать $[X]$.}
	 


 	 Для любого метрического пространства $X$ определена операция умножения его
на положительное вещественное число $\lambda\colon X\mapsto \lambda X$, а именно
$(X, \rho) \mapsto (X, \lambda \rho)$, расстояние между любыми точками
пространства изменяется в $\lambda$ раз.
 	 \begin{remark} Пусть метрические пространства $X$, $Y$ лежат в одном
облаке. Тогда $d_{GH}(\lambda X, \lambda Y) = \lambda d_{GH}(X,Y) < \infty$,
т.е. пространства $\lambda X$, $\lambda Y$ также будут лежать в одном облаке.
 	 \end{remark}  \Def{Определим операцию умножения облака $[X]$ на
положительное вещественное число $\lambda$ как отображение, переводящее все
пространства $Y \in [X]$ в пространства $\lambda Y$. По замечанию 1.2 все
полученные пространства будут лежать в облаке $[\lambda X]$.}
При таком отображении облако может как измениться, так и перейти в себя. Для последнего случая вводится специальное определение.
\Def{\emph{Стационарной группой}
$\St\bigl([X]\bigr)$ облака $[X]$ называется подмножество $\mathbb{R}_+$ такое,
что для всех $\lambda \in \St\bigl([X]\bigr)$, $[X] = [\lambda X]$. Полученное
подмножество действительно будет подгруппой в $\mathbb{R}_+$ (\cite{TuzhBog2}).}

Приведем несколько примеров облаков и их стационарных групп.
 	 
 	 \begin{itemize}
 	 	\item Пусть $\Delta_1$ --- одноточечное метрическое пространство.
Тогда\\ $\St\bigl([\Delta_1]\bigr) = \mathbb{R}_+$.
 	 	\item $\St\bigl([\mathbb{R}]\bigr) = \mathbb{R}_+ $.
 	 \end{itemize} \Def{(\cite{TuzhBog2}) Если стационарная группа некоторого
облака $[X]$ нетривиальна, то у него существует единственный \emph{центр}
$Z\bigl([X]\bigr)$ -- это такое метрическое пространство $Y \in [X]$, что
$d_{GH}(Y,\lambda Y) = 0$ для любых $\lambda \in \St \bigl([X]\bigr)$.}  Далее
за $R(X)$ будем обозначать образ пространства $X$ при
соответствии $R$, то есть
$R(X) = \{Y \in [\mathbb{R}] | (X, Y) \in R\}$. Под образом пространства будет
иметься ввиду именно образ при соответствии.  Аналогично определим $R^{-1}(Y)$.

 Для любых пространств $Y_1, Y_2 \in R(X)$ выполнено неравенство
$|Y_1, Y_2| = \big| |Y_1,Y_2| - |X,X|\big| \le \dis R$, откуда следует, что
диаметр  $R(X)$ не превосходит $\dis R$. Из этого, в частности, следует, что
пространства, лежащие на расстоянии большем, чем искажение соответствия, не
могут принадлежать образу или прообразу одного пространства.\\ По определению
искажения, если расстояние между пространствами $X_1, X_2$ равно $\rho$, то
расстояние между пространствами в их образах будет отличаться от $\rho$ не более
чем на $\dis R$. Это же верно и для пространств в прообразах.
 \begin{remark} В облаке $[\Delta_1]$ для любого пространства $X$ выполняется:
	$$|\lambda X, \mu X| = |\lambda - \mu||X,\Delta_1|.$$
 \end{remark}
 \begin{remark}[Ультраметрическое неравенство] В облаке $[\Delta_{1}]$ для всех пространств $X_{1}, X_{2}$
выполняется неравенство:
   $$|X_{1},X_{2}| \le \max\{|X_{1}, \Delta_{1}|,|X_{2},\Delta_{1}|\}$$
 \end{remark}

    %!TEX root = ../kursovaya.tex

\section{Теорема об образе центра}
\begin{theorem}
Пусть $M$ -- центр облака $[M]$, имеющего нетривиальную
\\стационарную группу. $R$ -- соответствие между $[\Delta_{1}]$ и $[M]$ с конечным
искажением $\epsilon$. Тогда образ пространства $\Delta_{1}$ лежит от $M$ на
расстоянии не большем $2\epsilon$.
\end{theorem}
\begin{proof}
Нетривиальность стационарной группы $[M]$ означает, что найдется
число $l > 1$ такое, что $\{l^{j}|j\in \mathbb{Z}\}$ является подгруппой в
$\St{[M]}$.\\
Зафиксируем $Y$ из образа $\Delta_{1}$.
Предположим, что $|M, Y| = d > \epsilon$.  Обозначим
$|Y, kY| = \rho$, $k \ge 2$, $k = l^{j_{1}}$. По неравенству треугольника $\rho + d \ge kd$,
откуда $\rho \ge (k-1)d > (k-1)\epsilon$. Тогда $kY$ лежит в образе
$X \ne \Delta_1$. При этом,
$\rho - \epsilon \le |X, \Delta_1| \le \rho + \epsilon$. \\
Возьмем произвольные $\alpha > 0$ и $\beta \in (0,1)$. Для пространств $(1+\alpha)X, (1-\beta)X$ будут выполняться неравенства:
	$$|X, (1+\alpha)X| = \alpha |X, \Delta_1| \le \alpha\rho + \alpha\epsilon,$$
	$$|X, (1-\beta)X| = \beta|X, \Delta_1| \le \beta\rho + \beta\epsilon,$$
	$$|(1+\alpha) X, (1-\beta)X| = (\alpha + \beta)|X, \Delta_1| \ge (\alpha+\beta)\rho - (\alpha+\beta)\epsilon.$$
 Существуют
$Y_\alpha, Y_\beta \in [M]$ такие, что
$kY_\alpha \in R\big((1+\alpha)X\big)$, $kY_\beta \in R\big((1-\beta)X\big)$, и
для них выполняются следующие неравенства:
	$$|kY, kY_\alpha| \le |X, (1+\alpha)X| + \epsilon \le \alpha\rho + (\alpha+1)\epsilon,$$
	$$|kY, kY_\beta| \le |X, (1-\beta)X| + \epsilon \le \beta\rho + (\beta+1)\epsilon,$$
	$$|kY_\alpha, kY_\beta| \ge |(1+\alpha)X, (1-\beta)X| - \epsilon \ge  (\alpha+\beta)\rho - (\alpha+\beta+1)\epsilon.$$
	Поделим эти неравенства на $k$:
	$$|Y, Y_{\alpha}| \le \frac{\alpha}{k}\rho + \frac{\alpha+1}{k}\epsilon,$$
	$$|Y, Y_{\beta}| \le \frac{\beta}{k}\rho + \frac{\beta+1}{k}\epsilon,$$
	$$|Y_\alpha, Y_{\beta}| \ge \frac{\alpha+\beta}{k}\rho - \frac{\alpha+\beta+1}{k}\epsilon.$$
	и возьмем прообразы пространств $Y, Y_{\alpha}, Y_{\beta}$:
	$$|\Delta_1, X_{\alpha}| \le \frac{\alpha}{k}\rho + \big(\frac{\alpha+1}{k} + 1\big)\epsilon,$$
	$$|\Delta, X_{\beta}| \le \frac{\beta}{k}\rho + \big(\frac{\beta+1}{k}+1\big)\epsilon,$$
	$$|X_\alpha, X_{\beta}| \ge \frac{\alpha+\beta}{k}\rho - \big(\frac{\alpha+\beta+1}{k}+1\big)\epsilon.$$
	 Считая, что $\alpha > \beta$ получаем неравенство:
	 $$\frac{\alpha+\beta}{k}\rho - \big(\frac{\alpha+\beta+1}{k}+1\big)\epsilon \le \frac{\alpha}{k}\rho + \big(\frac{\alpha+1}{k} + 1\big)\epsilon,$$
	 $$\Updownarrow$$
	 $$\rho \le \frac{k}{\beta}\bigg(\frac{2\alpha+\beta+2}{k}+2\bigg)\epsilon,$$
	 $$\Updownarrow$$
	 $$\rho \le \bigg(1+\frac{2\alpha + 2}{\beta} + 2\frac{k}{\beta}\bigg)\epsilon.$$
	Нас интересует оценка сверху для $d$:
	$$d \le \frac{\rho}{k-1} \le \bigg(\frac{1}{k-1}+\frac{2\alpha + 2}{\beta(k-1)} + 2\frac{k}{\beta(k-1)}\bigg)\epsilon $$
	Последнее слагаемое в скобках строго больше 2 при любых $k>2$, $\alpha>0$,
$\beta\in (0,1)$, а остальные слагаемые с ростом $k$ стремятся к $0$. Так как стационарная группа нетривиальна, в ней есть последовательности чисел стремящихся к 0 и к $\infty$.
Устремив $\beta$ к 1, а $k$ к бесконечности получаем оценку:
	$$|Y, M| \le 2\epsilon.$$

\end{proof}

    %!TEX root = ../kursovaya.tex

\section{Невыполнение ультраметрического неравенства}
Для облака $[\Delta_{1}]$ справедливо ультраметрическое неравнество(Замечание
1.4). Следующая лемма показывает, что для облака $[\mathbb[R]]$
это неравенство может не выполняться. \\
Рассмотрим $\mathbb{R}$ как подмножество
$\mathbb{R}^2$ и добавим к нему точку $(0,1)$, расстояние до которой будет
соответствовать манхэттенской метрике в $\mathbb{R}^2$. Обозначим это
пространство $\widetilde{\mathbb{R}}$.
\begin{theorem}
\begin{enumerate}
\itemПространства $\mathbb{Z}$ и $\widetilde{\mathbb{R}}$ находятся от $\mathbb{R}$
на расстоянии\\ не большем $\frac 1 2$.
\item Расстояние между $\mathbb{Z}$ и $\widetilde{\mathbb{R}}$ строго
больше $\frac 1 2$.
\end{enumerate}
\end{theorem}
\begin{proof}
Вложением целых чисел в вещественную прямую получается реализация $\mathbb{Z}$, $\mathbb{R}$ с расстоянием Хаусдорфа равным $\frac 1 2$.
Если вложить $\widetilde{\mathbb{R}}$ в $\mathbb{R}^{2}$ естественным образом, а $\mathbb{R}$ вложить как подмножество равное $\{(x, \frac1 2 )|x\in \mathbb{R}\}$,
расстояние Хаусдорва между ними также будет равно $\frac 1 2$. Таким образом, первое утверждение теоремы доказано. \\
Пусть $R$ --- соответствие
между $\mathbb{Z}$ и $\widetilde{\mathbb{R}}$, $\dis R < 1 + \epsilon$ и
$\big(i, (0,1)\big) \in R$. Тогда, если
$x \in \mathbb{R}\backslash(-\epsilon, \epsilon)$, то $(i, x) \notin R$.
Обозначим за $\mathcal{N}$ множество всех $k \in \mathbb{Z}$ таких, что для всех
$x \in \mathbb{R}\backslash(-\epsilon, \epsilon)$, $(k,x) \notin R$.
$\mathcal{N}$ не пусто и не равно $\mathbb{Z}$, следовательно, по лемме 3.1,
расстояние от $\mathbb{Z} \backslash \mathcal{N}$ до $\mathbb{R}$ будет не
меньше 1. Рассмотрим подмножество соответствия $R$,
$R' := R \backslash \{(k,x): x\in (-\epsilon, \epsilon), k \in \mathbb{Z}\} \cup \{\big(i, (0,1)\big)\}$.
Так как все точки из $\mathbb{R}\backslash(-\epsilon, \epsilon)$ лежат в $R$
только в паре с точками из $ \mathbb{Z} \backslash \mathcal{N}$ и наоборот,
множество $R'$ будет соответствием между
$\mathbb{R}\backslash(-\epsilon, \epsilon)$ и
$ \mathbb{Z} \backslash \mathcal{N}$  Тогда,
$\dis R \ge \dis R' \ge 2d_{GH}(\mathbb{R}\backslash(-\epsilon, \epsilon), \mathbb{Z} \backslash \mathcal{N})$
По неравенству треугольника
$2d_{GH}(\mathbb{R}\backslash(-\epsilon, \epsilon), \mathbb{Z} \backslash \mathcal{N}) \ge 2|d_{GH}(\mathbb{R}, \mathbb{Z} \backslash \mathcal{N}) - d_{GH}(\mathbb{R}\backslash(-\epsilon, \epsilon), \mathbb{R})| \ge 2 - 2\epsilon$,
что при малых $\epsilon$ больше, чем $1 + \epsilon$. Получаем, что
$d_{GH}(\widetilde{\mathbb{R}}, \mathbb{Z}) > \frac{1}{2}$.
\end{proof}

    %!TEX root = ../kursovaya.tex

\section{Подсчет расстояния между облаками $[\Delta_1]$ и $[\mathbb{R}]$}
\begin{theorem} Расстояние между облаками $[\Delta_1]$ и $[\mathbb{R}]$ равно
бесконечности.
\end{theorem} 
\begin{proof} У облаков $[\Delta_1]$ и $[\mathbb{R}]$ стационарные группы имеют
нетривиальное пересечение, и, по следствию 1, расстояние между ними может быть
равно либо $0$, либо $\infty$. \\ Для доказательства утверждения теоремы
достаточно будет показать, что расстояние между ними не равно $0$.  Для этого
необходимо установить, что между ними не может существовать соответствия со
сколь угодно малым искажением. Итак, пусть $R$ --- соответствие между
$[\Delta_1]$ и $[\mathbb{R}]$, $\dis R = \epsilon < \infty$.
Зафиксируем $Y$ из $R(\Delta_1)$. По теореме 2.1 расстояние между $Y$ и $\mathbb{R}$ не больше $2\epsilon$.\\  По теореме 3.1 пространства $\mathbb{Z}$, $\widetilde{\mathbb{R}}$ лежат от $\mathbb{R}$ на расстоянии не большем, чем $\frac{1}{2}$. При этом,
расстояние между этими пространствами строго больше $\frac{1}{2}$. Обозначим за
$r$ максимум из расстояний от этих пространств до $\mathbb{R}$:
	$$r = \max\big\{|\mathbb{Z}, \mathbb{R}|,|\widetilde{\mathbb{R}}, \mathbb{R}|\big\}\le \frac{1}{2} < |\mathbb{Z}, \widetilde{\mathbb{R}}|.$$
	Неравенство означает, что существует $c > 0$ такое, что
$|\mathbb{Z}, \widetilde{\mathbb{R}}| = (1 + c)r.$ \\ Вместе с $\mathbb{Z}$ и
$ \widetilde{\mathbb{R}}$ рассмотрим их прообразы $X_1 \in R^{-1}(\mathbb{Z})$,
$ X_2 \in R^{-1}(\widetilde{\mathbb{R}})$.  \\ Получаем следующую цепочку
неравенств:
	$$|X_1, \Delta_1| \le |\mathbb{Z}, Y| + \epsilon \le |\mathbb{Z}, \mathbb{R}| + |\mathbb{R}, Y| +\epsilon \le r + 2\epsilon + \epsilon = r + 3\epsilon.$$
	Аналогичное неравенство имеет место для $X_2$, при этом
	$$|X_1, X_2|  \ge |\mathbb{Z}, \widetilde{\mathbb{R}}| - \epsilon = (1+c)r - \epsilon.$$
	По замечанию 1.4:
	$$|X_1, X_2| \le \max\big\{ |X_1, \Delta_1|, |X_2, \Delta_1| \big\},$$
	$$\Updownarrow$$
	$$(1+c)r - \epsilon\le r + 3\epsilon,$$
	$$\Updownarrow$$
	$$\epsilon \ge \frac{cr}{4}.$$
	Мы получаем оценку снизу для $\epsilon = \dis R$. Это означает, что
искажение не может быть произвольно малым, и следовательно расстояние между
пространствами не может быть равно 0. Значит, оно равно бесконечности.
	
\end{proof}

}


% Информация о годе выполнения работы
\def\Year{%
    % 2006%
    \the\year%     % Текущий год
}

% Укажите тип работы
% Например:
%     Выпускная квалификационная работа,
%     Магистерская диссертация,
%     Курсовая работа, реферат и т.п.
\def\WorkType{%
    % Выпускная квалификационная работа%
    % Магистерская диссертация%
     Курсовая работа%
    % Реферат%
    %Дипломная работа%
}

% Название работы
%%%%%%%%%%% ВНИМАНИЕ! %%%%%%%%%%%%%%%%
% В МГУ ОНО ДОЛЖНО В ТОЧНОСТИ
% СООТВЕТСТВОВАТЬ ВЫПИСКЕ ИЗ ПРИКАЗА
% УТОЧНИТЕ НАЗВАНИЕ В УЧЕБНОЙ ЧАСТИ
\def\Title{%
   Подсчет расстояния Громова-Хаусдорфа между облаком ограниченных пространств и облаком с действительной прямой
}
\def\EngTitle{
  Calculating the Gromov-Hausdorff distance between the cloud of bounded metric spaces and the cloud containing the real line
}

% Имя автора работы
\def\Author{%
    Нестеров Борис Аркадьевич%
}

% Информация о научном руководителе
%% Фамилия Имя Отчество%
\def\SciAdvisor{%
    Тужилин Алексей Августинович%
}
%% В формате: И.~О.~Фамилия%
\def\SciAdvisorShort{%
    А.~А.~Тужилин%
}
%% должность научного руководителя
\def\Position{%
     профессор%
    %доцент%
    % старший преподаватель%
    % преподаватель%
    % ассистент%
    % ведущий научный сотрудник%
    % старший научный сотрудник%
    % научный сотрудник%
    % младший научный сотрудник%
}
%% учёная степень научного руководителя
\def\AcademicDegree{%
     д.ф.-м.н.%
    % д.т.н.%
    %к.ф.-м.н.%
    % к.т.н.%
    % без степени%
}

% Информация об организации, в которой выполнена работа
%% Город
\def\Place{%
    Москва%
}
%% Университет
\def\Univer{%
    Московский государственный университет имени М.~В.~Ломоносова%
}
%% Факультет
\def\Faculty{%
    Механико-математический факультет%
}
%% Кафедра    
\def\Department{%
    Кафедра дифференциальной геометрии и приложений%
}     

%%%% Переключите статус документа для отладки
%%%% В режиме draft документ собирается очень быстро
%%%% и выводится полезная информация о том
%%%% какие строки вылезают за границы документа, что удобно для борьбы с ними
\def\Status{%
     %draft%
    final%
}

%%%% Включает и выключает подпись <<С текстом работы ознакомлен>>
\def\EnableSign{%
    % true%
}
