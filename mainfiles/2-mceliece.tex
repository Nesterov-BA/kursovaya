  %!TEX root = ../kursovaya.tex

\section{Основные определения и предварительные результаты}
	Пусть $X$ и $Y$ --- метрические пространства. Тогда между ними можно задать расстояние, называемое расстоянием Громова--Хаусдорфа. Введем два? его эквивалентных (\cite{Lectures}) определения.
	\Def{Пусть $X$, $Y$ --- метрические пространства с метриками $\rho_X$ и $\rho_Y$ . \emph{Соответствием} $R$ между этими пространствами называется подмножество декартового произведения $X\times Y$ такое, что проекторы $\pi_X \colon (x,y) \mapsto x$, $\pi_Y\colon (x,y) \mapsto y$ являются сюрьективными. Множество всех соответствий между $X$ и $Y$ обозначается $\mathcal{R}(X,Y).$}
	\Def{ Пусть $R$ --- соответствие между $X$ и $Y$. \emph{Искажением} соответствия $R$, $\dis{R}$ является величина $$ \dis{R} = \sup{\Bigl\{ \big| |xx'| - |yy'| \big| : (x, y), (x', y') \in R\Bigr\}}.$$ Тогда \emph{расстояние Громова--Хаусдорфа} $d_{GH}(X,Y)$ можно определить следующим образом $$ d_{GH}(X,Y) = \frac{1}{2}\inf \bigl\{\dis{R} : R \in \mathcal{R}(X,Y)\bigr\}.$$}
	
	\Def{\emph{Реализацией} пары метрических пространств $(X,Y)$ назовем тройку метрических пространств $(X',Y',Z)$ таких, что $X' \subset Z$, $Y' \subset Z$, $X'$ изометрично $X$, $Y'$ изометрично $Y$. \emph{Расстояним Громова-Хаусдорфа} $d_{GH}(X,Y)$ между метрическими пространствами $X, Y$ является точная нижняя грань чисел $r$ таких, что существует реализация $(X',Y',Z)$ и \\$d_H(X', Y') \le r$, где $d_H$ --- расстояние Хаусдорфа.}
	
	Рассмотрим собственный класс всех метрических пространств и отождествим в нем между собой все метрические пространства, находящиеся на нулевом расстоянии друг от друга. Обозначим получившийся класс $\mathcal{GH}_0$. На нем расстояние Громова -- Хаусдорфа будет являться обобщенной метрикой.
	
	\Def{(\cite{TuzhBog1}) В классе $GH$ рассмотрим следующее отношение: $X \thicksim Y \Leftrightarrow d_{GH}(X, Y) < \infty$. Нетрудно убедиться, что оно будет отношением эквивалентности. Классы этой эквивалентности называются \emph{облаками}. Облако, в котором лежит метрическое пространство $X$ будем обозначать $[X]$.}
	 
	\begin{theorem}
		Все облака представляют собой собственные классы.
	\end{theorem}
	\begin{proof}
		Для доказательства теоремы достаточно показать, что в любом облаке лежат пространства сколь угодно большой мощности.
		Пусть $X$ -- метрическое пространство мощности $\alpha$. Расширим это пространство до пространства большей мощности. Обозначим $\Delta^\beta_1$ --- симплекс мощности $\beta$, где $\beta > \alpha$. Обозначим $X^\beta = X \cup \Delta^\beta_1$. Зафиксируем произвольную точку $x$ пространства $X$ и положим расстояние от нее до любой точки симплекса равным $1$. Для точек $x' \in X$, $y \in \Delta^\beta_1$ определим $\rho_{X_\beta}(x',y) := \rho_X(x',x) + 1$. Расстояния между другими парами точек оставим без изменений. Для того, чтобы полученное расстояние являлось метрикой достаточно проверить выполнение неравенства треугольника $\rho_{X_\beta}(x',z') \le \rho_{X_\beta}(x',y') +\rho_{X_\beta}(y',z')$  
		только в том случае, если точки $x', y', z'$ не лежат одновременно в $\Delta^\beta_1$ или в $X$. Случаи $x', z' \in \Delta^\beta_1$ и $ x', z' \in X$ очевидны. Разберем подробнее случаи, когда $x' \in X, z' \in \Delta^\beta_1$:
		$$ y' \in X: \rho_{X_\beta}(x', z') = \rho_X(x,x') + 1 \le \rho_X(x,y') + \rho_X(y',x') + 1 = \rho_X(x',y') + \rho_X(y',z')$$
		$$y' \in \Delta^\beta_1: \rho_{X_\beta}(x', z') = \rho_X(x,x') + 1 \le \rho_X(x',x) + 2 = \rho_X(x',y') + \rho_X(y',z')$$
		Итак, полученное пространство действительно будет метрическим. Осталось заметить, что если вложить $X$ в $X^\beta$, то $X^\beta$ будет лежать в замкнутой окрестности $X$ радиуса 1, что означает конечность расстояния между ними.
 	 \end{proof}
 	 
 	 \begin{remark}
 	 	Поскольку все облака являются собственными классами, между любыми двумя облаками существует биекция.
 	 \end{remark}
 	 
 	 \Def{Пусть $\mathcal{R}\big([X],[Y]\big)$ --- класс всех биекций между облаками $[X]$ и $[Y]$. Определим \emph{искажение} соответствия $\dis R$ аналогично определению 1.2. \emph{Расстоянием Громова--Хаусдорфа} между облаками будем называть величину $d_{GH}\big([X],[Y]\big) = \frac{1}{2}\inf\Big(\dis R : R\in \mathcal{R}\big([X],[Y]\big)\Big)$.}
 	 
 	 Для любого метрического пространства $X$ определена операция умножения его на положительное вещественное число $\lambda\colon X\mapsto \lambda X$, а именно $(X, \rho) \mapsto (X, \lambda \rho)$, расстояние между любыми точками пространства изменяется в $\lambda$ раз.
 	 \begin{remark}
 	 	Пусть метрические пространства $X$, $Y$ лежат в одном облаке. Тогда $d_{GH}(\lambda X, \lambda Y) = \lambda d_{GH}(X,Y) < \infty$, т.е. пространства $\lambda X$, $\lambda Y$ также будут лежать в одном облаке.
 	 \end{remark} 
 	 \Def{Определим операцию умножения облака $[X]$ на положительное вещественное число $\lambda$ как отображение, переводящее все пространства $Y \in [X]$ в пространства $\lambda Y$. По замечанию 2.1 все полученные пространства будут лежать в облаке $[\lambda X]$.}
 	 Особый интерес представляет случай, когда такое отображение оказывается тождественным, в связи с чем вводится следующее определение.
 	 \Def{\emph{Стационарной группой} $\St\bigl([X]\bigr)$ облака $[X]$ называется подмножество $\mathbb{R}_+$ такое, что для всех $\lambda \in \St\bigl([X]\bigr)$, $[X] = [\lambda X]$. Полученное подмножество действительно будет подгруппой в $\mathbb{R}_+$ (\cite{TuzhBog2}).}
 	 
 	 Приведем несколько примеров облаков и их стационарных групп.
 	 
 	 \begin{itemize}
 	 	\item Пусть $\Delta_1$ --- одноточечное метрическое пространство. Тогда\\ $\St\bigl([\Delta_1]\bigr) = \mathbb{R}_+$.
 	 	\item $\St\bigl([\mathbb{R}]\bigr) = \mathbb{R}_+ $.
 	 \end{itemize}
 	 \Def{(\cite{TuzhBog2}) Если стационарная группа некоторого облака $[X]$ нетривиальна, то у него существует единственный \emph{центр} $Z\bigl([X]\bigr)$ -- это такое метрическое пространство $Y \in [X]$, что $d_{GH}(Y,\lambda Y) = 0$ для любых $\lambda \in \St \bigl([X]\bigr)$.}
 	 
 	 
	