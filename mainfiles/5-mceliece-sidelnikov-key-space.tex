%!TEX root = ../kursovaya.tex

\section{Подсчет расстояния между облаками $[\Delta_1]$ и $[\mathbb{R}]$}
\begin{theorem}
	Расстояния между облаками $[\Delta_1]$ и $[\mathbb{R}]$ равно бесконечности.
\end{theorem} 

\begin{proof}
	Так как у облаков $[\Delta_1]$ и $[\mathbb{R}]$ совпадают стационарные группы, по следствию 1 расстояние между ними может быть равно либо $0$, либо $\infty$. Покажем, что оно не равно $0$. Для этого необходимо установить, что между ними не может существовать соответствия со сколь угодно малым искажением. Итак, пусть $R$ --- соответствие между $[\Delta_1]$ и $[\mathbb{R}]$, $\dis R = \epsilon$.
	
	$\{X_\alpha\}$ --- $\epsilon$-сеть в $B(\Delta_1,1)$ такая, что $|X_{\alpha_1}, X_{\alpha_2}| > \epsilon$ для любых $\alpha_1, \alpha_2$. Значит за $P(X_\alpha)$ множество всех $Y\in \mathbb{R}$ таких, что $(X_\alpha, Y)\in R$.
	
	Если $Y_1, Y_2 \in P(X_\alpha)$, то $|Y_1, Y_2| \le \epsilon$. Если $Y \in P(X_{\alpha_1}) \cap P(X_{\alpha_2})$, то $|X_{\alpha_1}, X_{\alpha_2}| \le \epsilon$, что противоречит выбору $X_\alpha$. Следовательно для любых $\alpha_1, \alpha_2$, $P(X_{\alpha_1})$ не пересекается с $P(X_{\alpha_2})$, и в каждом $P(X_\alpha)$ можно выбрать уникальный $Y_\alpha$. Далее под $P(X_\alpha)$ будем подразумевать именно этот $Y_\alpha$.
	\begin{lemma}
		$P(\Delta_1)$ лежит от $\mathbb{R}$ на расстоянии, не большем $2\epsilon$.
	\end{lemma}
	\begin{proof}
		Если $P(\Delta_1)$ не совпадает с $\mathbb{R}$, то через $P(\Delta_1)$ проходит геодезическая, и существуют пространства $Y_1, Y_2 \in [\mathbb{R}]$ такие, что $|Y_1,P(\Delta_1)| = |Y_2,P(\Delta_1)| = \rho$, $|Y_1,Y_2| = 2\rho$, причем $\rho$ можно взять сколь угодно б. Существуют $X_1, X_2\in [\Delta_1]$, лежащие в соответствии $R$ в паре с $Y_1, Y_2$ соответственно. Тогда для них верны следующие неравенства 
		$$\rho - \epsilon \le |X_1, \Delta_1| \le \rho + \epsilon$$
		$$\rho - \epsilon \le |X_2, \Delta_1| \le \rho + \epsilon$$
		Из этих неравенств получаем, что $\rho - \epsilon \le |X_1, X_2| \le \rho + \epsilon$. Вычитая последнее неравенство из расстояния между $|Y_1, Y_2|$ получаем $$\rho - \epsilon \le |Y_1,Y_2| - |X_1,X_2|\le \rho + \epsilon$$
		При этом из того, что искажение соответствия $R$ равно $\epsilon$ следует
		$$\big||X_1,X_2| - |Y_1, Y_2|\big| \le \epsilon$$ 
		Если $\rho > \epsilon$, то из последних двух неравенств следует, что $\rho < 2\epsilon$. Так как $\rho$ можно выбрать сколь угодно близко к $|\mathbb{R}, P(\Delta_1)|$, получаем, что $|\mathbb{R}, P(\Delta_1)| \le 2\epsilon$.
	\end{proof}
	Продолжим доказательство теоремы.
	
	Если взять $B(\mathbb{R}, r)$, то начиная с некоторого $\epsilon$ $\{P(X_\alpha)\}$ будет $2\epsilon$- сетью в $B(\mathbb{R}, r)$. Возьмем пространства $Y_1, Y_2\in B(\mathbb{R}, r)$ такие, что $|Y_1,\mathbb{R}| = |Y_2,\mathbb{R}| = \rho$, $|Y_1,Y_2| = 2\rho$. Так как $\{P(X_\alpha)\}$, существуют такие $X_{\alpha_1}, X_{\alpha_2}\in [\Delta_1]$ для которых $|Y_1,P(X_{\alpha_1})| \le 2\epsilon, |Y_2,P(X_{\alpha_2})| \le 2\epsilon$. Из этих неравенств получаем следующее
	
	$$|P(X_{\alpha_1}), P(X_{\alpha_2})| \ge 2\rho - 4\epsilon$$
	$$|P(X_{\alpha_i}), P(\Delta_1)| \le \rho + 4\epsilon, i = 1, 2$$
	
	Так, как $P$ -- биекция, можно взять обратное отображение $P^{-1}$, при этом расстояния не могут измениться больше, чем на $\epsilon$. $\rho$ не зависит от $\epsilon$, и при $\epsilon \ll \rho < 1$ получим, что система неравенств выше не выполняется. Получили противоречие, следовательно соответствие со сколь угодно малым искажением не может существовать. Значит расстояние между облаками $[\Delta_1]$ и $[\mathbb{R}]$ может быть равно только бесконечности, что завершает доказательство.
\end{proof}