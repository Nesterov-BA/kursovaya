%!TEX root = ../kursovaya.tex

\section{Случай $[\Delta_1]$ и $[\mathbb{R}]$}

Вопрос об установлении точного расстояния Громова--Хаусдорфа между метрическими пространствами часто сопровождается вопросом об изометрии между этими пространствами. Известно, что изометричные пространства лежат на нулевом расстоянии друг от друга. Обратное вообще говоря не верно.

Мы рассмотрим вопрос изометрии конкретных облаков, а именно $[\Delta_1]$ и $[\mathbb{R}]$, так как эти пространства обладают рядом свойств, позволяющих значительно упростить задачу по поиску изометрии между ними.

\begin{remark}
	Центром облака $[\Delta_1]$ является одноточечный симплекс $\Delta_1$. Пусть $d_{GH}(\Delta_1, X) < l$, $d_{GH}(\Delta_1, Y) < l$, равносильно $\diam X < 2l$, $\diam Y < 2l$. Тогда $d_{GH}(X,Y) < l$ (\cite{Lectures}). Это означает, что шар радиуса $l$ с центром в $\Delta_1$ имеет диаметр $l$. 
\end{remark}

Открытый шар с центром в $x$ радиуса $d$ будем обозначать $B(x,d)$.
\begin{lemma}
	Пусть $X$ --- подмножество $\mathbb{R}$ такое, что в $\mathbb{R} \backslash X$ лежит интервал длиной $2d$. Тогда пространство $X$ лежит от $\mathbb{R}$ на расстоянии не меньшем, чем $d$.
\end{lemma}

\begin{proof}
	В $\mathbb{R} \backslash X$ лежит интервал $(a-d,a+d)$. Предположим, что $d_{GH}(\mathbb{R}, X) < d$. Пусть $(\mathbb{R}', X', Y)$ --- реализация $(\mathbb{R}, X)$ такая, что $d_H(\mathbb{R}', X') = d' < d$. Обозначим $U_1 := \cup_{x \in X', x \le a-d}B(x, d'+\frac{d-d'}{2})$, $U_2 := \cup_{x \in X', x \ge a+d}B(x, d'+\frac{d-d'}{2})$. Получаем, что $U_1, U_2$ --- два открытых непересекающихся множества, но также $\mathbb{R}' \in U_1\cup U_2$, что противоречит связности прямой.
\end{proof}

\begin{theorem}
	Образ $\Delta_1$ при изометрии не может равняться $\mathbb{R}$.
\end{theorem} 

\begin{proof}
	Для доказательства леммы достаточно предъявить два \\метрических пространства, лежащих друг от друга на расстоянии большем, чем максимум их расстояний до $\mathbb{R}$. Рассмотрим $\mathbb{R}$ как подмножество $\mathbb{R}^2$ и добавим к нему точку $(0,1)$, расстояние до которой будет соответствовать манхэттенской метрике в $\mathbb{R}^2$. Обозначим это пространство $\widetilde{\mathbb{R}}$, тогда $d_{GH}(\mathbb{R}, \widetilde{\mathbb{R}}) \le \frac{1}{2}$. Также $d_{GH}(\mathbb{R},\mathbb{Z}) \le \frac{1}{2}$. Пусть $R$ --- соответствие между $\mathbb{Z}$ и $\widetilde{\mathbb{R}}$, $\dis R < 1 + \epsilon$ и $\big(i, (0,1)\big) \in R$. Тогда, если $x \in \mathbb{R}\backslash(-\epsilon, \epsilon)$, то $(i, x) \notin R$. Обозначим за $\mathcal{N}$ множество всех $k \in \mathbb{Z}$ таких, что для всех $x \in \mathbb{R}\backslash(-\epsilon, \epsilon)$, $(k,x) \notin R$. $\mathcal{N}$ не пусто и не равно $\mathbb{Z}$, следовательно, по лемме 3.1, расстояние от $\mathbb{Z} \backslash \mathcal{N}$ до $\mathbb{R}$ будет не меньше 1. Рассмотрим подмножество соответствия $R$, $R' := R \backslash \{(k,x): x\in (-\epsilon, \epsilon), k \in \mathbb{Z}\} \cup \{\big(i, (0,1)\big)\}$. Так как все точки из $\mathbb{R}\backslash(-\epsilon, \epsilon)$ лежат в $R$ только в паре с точками из $ \mathbb{Z} \backslash \mathcal{N}$ и наоборот, множество $R'$ будет соответствием между $\mathbb{R}\backslash(-\epsilon, \epsilon)$ и $ \mathbb{Z} \backslash \mathcal{N}$  Тогда, $\dis R \ge \dis R' \ge 2d_{GH}(\mathbb{R}\backslash(-\epsilon, \epsilon), \mathbb{Z} \backslash \mathcal{N})$ По неравенству треугольника $2d_{GH}(\mathbb{R}\backslash(-\epsilon, \epsilon), \mathbb{Z} \backslash \mathcal{N}) \ge 2|d_{GH}(\mathbb{R}, \mathbb{Z} \backslash \mathcal{N}) - d_{GH}(\mathbb{R}\backslash(-\epsilon, \epsilon), \mathbb{R})| \ge 2 - 2\epsilon$, что при малых $\epsilon$ больше, чем $1 + \epsilon$. Получаем, что $d_{GH}(\widetilde{\mathbb{R}}, \mathbb{Z}) > \frac{1}{2}$.    
\end{proof}

\begin{corollary}
	Облака $[\Delta_1]$ и $[\mathbb{R}]$ не изометричны.
\end{corollary}

\begin{proof}
	Предположим, что существует изометрия $F\colon$$[\Delta_1] \mapsto [\mathbb{R}]$. По предыдущей теореме получили, что $F(\Delta_1) = X \neq \mathbb{R}$. Рассмотрим пространство $\lambda X$, которое будет являться образом пространства $\widetilde{X}$ из $[\Delta_1]$, отличного от $\Delta_1$. Через $\widetilde{X}$ в облаке $[\Delta_1]$ проходит геодезическая, а значит через $\lambda X$ проходит образ этой геодезической. Это означает, что найдутся пространства $Y_1, Y_2$ такие, что $d_{GH}(\lambda X, Y_1) = d_{GH}(\lambda X, Y_2) = d$ и $d_{GH}(Y_1, Y_2) = 2d$. Поделим эти равенства на $\lambda: $ $d_{GH}(X, \frac{1}{\lambda} Y_1) = d_{GH}(X, \frac{1}{\lambda} Y_2) = \frac{d}{\lambda}$ и $d_{GH}(\frac{1}{\lambda} Y_1, \frac{1}{\lambda} Y_2) = \frac{2d}{\lambda}$. Теперь, если мы рассмотрим прообразы пространств $X, \frac{1}{\lambda} Y_1, \frac{1}{\lambda} Y_2$ при изометрии $F$, то получим противоречие с замечанием 3.1.
\end{proof}



