%!TEX root = ../kursovaya.tex

\section{Подсчет расстояния между облаками $[\Delta_1]$ и $[\mathbb{R}]$}
Далее за $R(X)$ будем обозначать образ пространства $X$ из $[\Delta_1]$ при соответствии $R$ между$[\Delta_1]$ и $[\mathbb{R}]$, то есть $R(X) = \{Y \in [\mathbb{R}] | (X, Y) \in R\}$. Под образом пространства будет иметься ввиду именно образ при соответствии.
 Аналогично определим $R^{-1}(Y)$. \\
 Для любых пространств $Y_1, Y_2 \in R(X)$ выполнено неравенство $|Y_1, Y_2| = \big| |Y_1,Y_2| - |X,X|\big| \le \dis R$, откуда следует, что диаметр  $R(X)$ не превосходит $\dis R$. Из этого, в частности, следует, что пространства, лежащие на расстоянии большем, чем искажение соответствия, не могут принадлежать образу или прообразу одного пространства.\\
 По определению искажения, если расстояние между пространствами $X_1, X_2$ равно $\rho$, то расстояние между пространствами в их образах будет отличаться от $\rho$ не более чем на $\dis R$. Это же верно и для пространств в прообразах.

\begin{theorem}
	Расстояние между облаками $[\Delta_1]$ и $[\mathbb{R}]$ равно бесконечности.
\end{theorem} 
\begin{proof}
	У облаков $[\Delta_1]$ и $[\mathbb{R}]$ стационарные группы имеют нетривиальное пересечение, и, по следствию 1, расстояние между ними может быть равно либо $0$, либо $\infty$. \\
	Для доказательства утверждения теоремы достаточно будет показать, что расстояние между ними не равно $0$.
	Для этого необходимо установить, что между ними не может существовать соответствия со сколь угодно малым искажением. Итак, пусть $R$ --- соответствие между $[\Delta_1]$ и $[\mathbb{R}]$, $\dis R = \epsilon < \infty$.
	
	\begin{lemma}
		Если $Y$ лежит в образе $\Delta_1$, то $Y$ лежит от $\mathbb{R} $ на расстоянии, не большем $15\epsilon$.
	\end{lemma}
	\begin{proof}
		При $|\mathbb{R}, Y| \le \epsilon$ утверждение леммы выполняется.
		Предположим, что $|\mathbb{R}, Y| = d > \epsilon$.
		Обозначим $|Y, 2Y| = \rho$, по неравенству треугольника $\rho + d \ge 2d$, откуда $\rho \ge d > \epsilon$. Тогда $2Y$ лежит в образе $X \ne \Delta_1$. При этом, $\rho - \epsilon \le |X, \Delta_1| \le \rho + \epsilon$. \\
		В облаке $[\Delta_1]$ для любого пространства $X$ выполняется:
		$$|\lambda X, \mu X| = |\lambda - \mu||X,\Delta_1|.$$
		Для $\frac{3}{2}X, \frac{1}{2}X$ получаем следующие неравенства:
		$$|X, \frac{1}{2}X| = \frac 1 2 |X, \Delta_1| \le \frac{\rho + \epsilon}{2},$$ 
		$$|X, \frac{3}{2}X| = \frac 1 2 |X, \Delta_1| \le \frac{\rho + \epsilon}{2},$$ 
		$$\big|\frac 1 2 X, \frac{3}{2}X\big| = |X, \Delta_1| \ge \rho - \epsilon$$ 
		Существуют $Y_1, Y_2 \in [\mathbb{R}]$ такие, что $2Y_1 \in R(\frac{1}{2}X)$, $2Y_2 \in R(\frac{3}{2}X)$, и для них выполняются следующие неравенства:
		$$|2Y, 2Y_{1}| \le |X, \frac{1}{2}X| + \epsilon \le \frac{\rho}{2} + \frac{3}{2}\epsilon,$$ 
		$$|2Y, 2Y_{2}| \le |X, \frac{3}{2}X| + \epsilon \le \frac{\rho}{2} + \frac{3}{2}\epsilon,$$
		$$|2Y_1, 2Y_{2}| \ge \big|\frac{1}{2}X, \frac{3}{2}X\big| - \epsilon \ge  \rho - 2\epsilon$$
		Поделим эти неравенства на 2:
		$$|Y, Y_{1}| \le \frac{\rho}{4} + \frac{3}{4}\epsilon,$$ 
		$$|Y, Y_{2}| \le \frac{\rho}{4} + \frac{3}{4}\epsilon,$$
		$$|Y_1, Y_{2}| \ge \frac{\rho}{2} - \epsilon$$ 
		\newpage
		\noindent и возьмем прообразы пространств $Y$, $Y_1$, $Y_2$:
		$$|\Delta_1, X_1| \le |Y, Y_{1}| + \epsilon \le \frac{\rho}{4} + \frac{7}{4}\epsilon,$$ 
		$$|\Delta_1, X_2| \le |Y, Y_{2}| + \epsilon \le \frac{\rho}{4} + \frac{7}{4}\epsilon,$$ 
		$$|X_1, X_2| \ge  |Y_1, Y_2| - \epsilon \ge \frac{\rho}{2} - 2\epsilon$$
		Итак ,по замечанию 3.1 необходимо выполнение следующего неравенства
		$$\frac{\rho}{4} + \frac{7}{4}\epsilon \ge \frac{\rho}{2} - 2\epsilon,$$
		что равносильно 
		$$\rho \le 15\epsilon,$$
		откуда получаем $15 \epsilon \ge \rho \ge |\mathbb{R}, Y|$
	\end{proof}
	Продолжим доказательство теоремы.

	Зафиксируем $Y$ из $R(\Delta_1)$. 
	В доказательстве теоремы 3.1 были построены пространства $\mathbb{Z}, \widetilde{\mathbb{R}}$ в облаке $[\mathbb{R}]$, лежащие от $\mathbb{R}$ на расстоянии не большем, чем $\frac{1}{2}$. При этом, расстояние между этими пространствами строго больше $\frac{1}{2}$. Обозначим за $r$ максимум из расстояний от этих пространств до $\mathbb{R}$:
	$$r = \max\big\{|\mathbb{Z}, \mathbb{R}|,|\widetilde{\mathbb{R}}, \mathbb{R}|\big\}\le \frac{1}{2} < |\mathbb{Z}, \widetilde{\mathbb{R}}|$$
	Неравенство означает, что существует $c > 0$ такое, что $|\mathbb{Z}, \widetilde{\mathbb{R}}| = (1 + c)r$
	\\
	Вместе с $\mathbb{Z}$ и $ \widetilde{\mathbb{R}}$ рассмотрим их прообразы $X_1 \in R^{-1}(\mathbb{Z})$, $ X_2 \in R^{-1}(\widetilde{\mathbb{R}})$.
	\\
	Получаем следующую цепочку неравенств:
	$$|X_1, \Delta_1| \le |\mathbb{Z}, Y| + \epsilon \le |\mathbb{Z}, \mathbb{R}| + |\mathbb{R}, Y| +\epsilon \le r + 15\epsilon + \epsilon = r + 16\epsilon$$
	Аналогичное неравенство имеет место для $X_2$, при этом
	$$|X_1, X_2|  \ge |\mathbb{Z}, \widetilde{\mathbb{R}}| - \epsilon = (1+c)r - \epsilon$$
	По замечанию 3.1:
	$$|X_1, X_2| \le \max\big\{ |X_1, \Delta_1|, |X_2, \Delta_1| \big\} $$
	$$(1+c)r - \epsilon\le r + 16\epsilon$$
	$$\epsilon \ge \frac{cr}{17}$$
	Мы получаем оценку снизу для $\epsilon = \dis R$. Это означает, что искажение не может быть произвольно малым, и следовательно расстояние между пространствами не может быть равно 0. Значит оно равно бесконечности.
	
\end{proof}

\begin{remark}
	В доказательстве леммы 4.1 из свойств облака $[\mathbb{R}]$ используется только вид его стационарной группы. Нетрудно заметить, что утверждение леммы без труда переносится на другие облака со стационарной группой $\mathbb{R}^+$, в частности на $[\mathbb{R}^n]$.
\end{remark}

\begin{remark}
	В доказательстве леммы 4.1 выбор коэффициентов при пространствах $X$ и $Y$ является произвольным, и представленная оценка может оказаться грубой. В следующей лемме эта оценка уточняется.
\end{remark}

\begin{lemma}
	Если $Y$ лежит в образе $\Delta_1$, то $Y$ лежит от $\mathbb{R} $ на расстоянии, не большем $2\epsilon$.
\end{lemma}
\begin{proof}
	Предположим, что $|\mathbb{R}, Y| = d > \epsilon$.
	Обозначим $|Y, kY| = \rho$, $k \ge 2$. По неравенству треугольника $\rho + d \ge kd$, откуда $\rho \ge (k-1)d > (k-1)\epsilon$. Тогда $kY$ лежит в образе $X \ne \Delta_1$. При этом, $\rho - \epsilon \le |X, \Delta_1| \le \rho + \epsilon$. \\
	Для $(1+\alpha)X, (1-\beta)X$ получаем следующие неравенства:
	$$|X, (1+\alpha)X| = \alpha |X, \Delta_1| \le \alpha\rho + \alpha\epsilon,$$ 
	$$|X, (1-\beta)X| = \beta|X, \Delta_1| \le \beta\rho + \beta\epsilon,$$ 
	$$|(1+\alpha) X, (1-\beta)X| = (\alpha + \beta)|X, \Delta_1| \ge (\alpha+\beta)\rho - (\alpha+\beta)\epsilon$$ 
	Отметим, что $\alpha$ может принимать любые положительные значения, а $\beta$ лежит в интервале $(0, 1)$.\\
	Существуют $Y_\alpha, Y_\beta \in [\mathbb{R}]$ такие, что $kY_\alpha \in R\big((1+\alpha)X\big)$, $kY_\beta \in R\big((1-\beta)X\big)$, и для них выполняются следующие неравенства:
	$$|kY, kY_\alpha| \le |X, (1+\alpha)X| + \epsilon \le \alpha\rho + (\alpha+1)\epsilon,$$ 
	$$|kY, kY_\beta| \le |X, (1+\beta)X| + \epsilon \le \beta\rho + (\beta+1)\epsilon,$$
	$$|kY_\alpha, kY_\beta| \ge |(1+\alpha)X, (1-\beta)X| - \epsilon \ge  (\alpha+\beta)\rho - (\alpha+\beta+1)\epsilon$$
	\newpage
	\noindent Поделим эти неравенства на $k$:
	$$|Y, Y_{\alpha}| \le \frac{\alpha}{k}\rho + \frac{\alpha+1}{k}\epsilon,$$ 
	$$|Y, Y_{\beta}| \le \frac{\beta}{k}\rho + \frac{\beta+1}{k}\epsilon,$$
	$$|Y_\alpha, Y_{\beta}| \ge \frac{\alpha+\beta}{k}\rho - \frac{\alpha+\beta+1}{k}\epsilon$$ 
	и возьмем прообразы этих пространств:
	$$|\Delta_1, X_{\alpha}| \le \frac{\alpha}{k}\rho + \big(\frac{\alpha+1}{k} + 1\big)\epsilon,$$ 
	$$|\Delta, X_{\beta}| \le \frac{\beta}{k}\rho + \big(\frac{\beta+1}{k}+1\big)\epsilon,$$ 
	$$|X_\alpha, X_{\beta}| \ge \frac{\alpha+\beta}{k}\rho - \big(\frac{\alpha+\beta+1}{k}+1\big)\epsilon$$ 
	 Считая, что $\alpha > \beta$ получаем неравенство:
	 $$\frac{\alpha+\beta}{k}\rho - \big(\frac{\alpha+\beta+1}{k}+1\big)\epsilon \le \frac{\alpha}{k}\rho + \big(\frac{\alpha+1}{k} + 1\big)\epsilon$$
	 $$\rho \le \frac{k}{\beta}\bigg(\frac{2\alpha+\beta+2}{k}+2\bigg)\epsilon$$
	 $$\rho \le \bigg(1+\frac{2\alpha + 2}{\beta} + 2\frac{k}{\beta}\bigg)\epsilon$$
	Нас интересует оценка сверху для $d$:
	$$d \le \frac{\rho}{k-1} \le \bigg(\frac{1}{k-1}+\frac{2\alpha + 2}{\beta(k-1)} + 2\frac{k}{\beta(k-1)}\bigg)\epsilon $$
	Нетрудно заметить, что последнее слагаемое в скобках строго больше 2 при любых $k$, $\alpha$, $\beta\in (0,1)$, а остальные слагаемые с ростом $k$ стремятся к $0$. Устремив $\beta$ к 1, а $k$ к бесконечности получаем оценку:
	$$|Y, \mathbb{R}| \le 2\epsilon$$
\end{proof}

