%!TEX root = kursovaya.tex
% Добавьте ссылку на файлы с текстом работы
% Можно использовать команды:
%   \input или \include
% Пример:
%    \input{mainfiles/1-section} или \include{mainfiles/2-section}
% Команда \input позволяет включить текст файла без дополнительной обработки
% Команда \include при включении файла добавляет до него и после него команду
% перехода на новую страницу. Кроме того, она позволяет компилировать каждый файл
% в отдельности, что ускоряет сборку проекта.
% ВАЖНО: команда \include не поддерживает включение файлов, в которых уже содержится команда \include,
% т.е. не возможен рекурсивный вызов \include
\newcommand*{\Source}{
    %!TEX root = ../kursovaya.tex
\phantomsection
\section*{Введение} 
\addcontentsline{toc}{section}{Введение}
	Понятие облаков метрических пространств является частью теории расстояния Громова--Хаусдорфа. Так как облака сами по себе являются метрическими классами, естественно ввести на собственном классе облаков обобщенную псевдометрику, являющуюся аналогом расстояния Громова-Хаусдорфа.
	В данной работе приведены некоторые результаты изучения ее свойств, а также поднимается вопрос об изометрическом изоморфизме различных облаков, тесно связанный с вопросом расстояния между ними. 
      %!TEX root = ../kursovaya.tex

\section{Основные определения и предварительные результаты}
	Пусть $X$ и $Y$ --- метрические пространства. Тогда между ними можно задать расстояние, называемое расстоянием Громова--Хаусдорфа. Введем два его эквивалентных (\cite{Lectures}) определения.
	\Def{Пусть $X$, $Y$ --- метрические пространства с метриками $\rho_X$ и $\rho_Y$ . \emph{Соответствием} $R$ между этими пространствами называется подмножество декартового произведения $X\times Y$ такое, что проекторы $\pi_X \colon (x,y) \mapsto x$, $\pi_Y\colon (x,y) \mapsto y$ являются сюрьективными. Множество всех соответствий между $X$ и $Y$ обозначается $\mathcal{R}(X,Y).$}
	\Def{ Пусть $R$ --- соответствие между $X$ и $Y$. \emph{Искажением} соответствия $R$ , $\dis{R}$ является величина $$ \dis{R} = \sup{\Bigl\{ \big| |xx'| - |yy'| \big| : (x, y), (x', y') \in R\Bigr\}}.$$ Тогда \emph{расстояние Громова--Хаусдорфа} $d_{GH}(X,Y)$ можно определить следующим образом $$ d_{GH}(X,Y) = \frac{1}{2}\inf \bigl\{\dis{R} : R \in \mathcal{R}(X,Y)\bigr\}.$$}
	
	\Def{\emph{Реализацией} пары метрических пространств $(X,Y)$ назовем тройку метрических пространств $(X',Y',Z)$ таких, что $X' \subset Z$, $Y' \subset Z$, $X'$ изометрично $X$, $Y'$ изометрично $Y$. \emph{Расстояним Громова-Хаусдорфа} $d_{GH}(X,Y)$ между метрическими пространствами $X, Y$ является точная нижняя грань чисел $r$ таких, что существует реализация $(X',Y',Z)$ и $d_H(X', Y') \le r$, где $d_H$ --- расстояние Хаусдорфа.}
	
	Рассмотрим собственный класс всех метрических пространств и отождествим в нем между собой все метрические пространства, находящиеся на нулевом расстоянии друг от друга. Обозначим получившийся класс $\mathcal{GH}_0$. На нем расстояние Громова -- Хаусдорфа будет являться обобщенной метрикой.
	
	\Def{(\cite{TuzhBog1}) В классе $GH$ рассмотрим следующее отношение: $X \thicksim Y \Leftrightarrow d_{GH}(X, Y) < \infty$. Нетрудно убедиться, что оно будет отношением эквивалентности. Классы этой эквивалентности называются \emph{облаками}. Пусть $X$ -- метрическое пространство. Облако, в котором лежит $X$ будем обозначать $[X]$.}
	 
	\begin{theorem}
		Все облака представляют собой собственные классы.
	\end{theorem}
	\begin{proof}
		Для доказательства теоремы достаточно показать, что в любом облаке лежат пространства сколь угодно большой мощности.
		Пусть $X$ -- метрическое пространство мощности $\alpha$. Расширим это пространство до пространства большей мощности. Обозначим $\Delta^\beta_1$ --- симплекс мощности $\beta$, где $\beta > \alpha$. Обозначим $X^\beta = X \cup \Delta^\beta_1$. Зафиксируем произвольную точку $x$ пространства $X$ и положим расстояние от нее до любой точки симплекса равным $1$. Для точек $x' \in X$, $y \in \Delta^\beta_1$ определим $\rho_{X_\beta}(x',y) := \rho_X(x',x) + 1$. Расстояния между другими парами точек оставим без изменений. Для того, чтобы полученное расстояние являлось метрикой достаточно проверить выполнение неравенства треугольника $\rho_{X_\beta}(x',z') \le \rho_{X_\beta}(x',y') +\rho_{X_\beta}(y',z')$  
		только в том случае, если точки $x', y', z'$ не лежат одновременно в $\Delta^\beta_1$ или в $X$. Случаи $x', z' \in \Delta^\beta_1$ и $ x', z' \in X$ очевидны. Разберем подробнее случаи, когда $x' \in X, z' \in \Delta^\beta_1$:
		$$ y' \in X: \rho_{X_\beta}(x', z') = \rho_X(x,x') + 1 \le \rho_X(x,y') + \rho_X(y',x') + 1 = \rho_X(x',y') + \rho_X(y',z')$$
		$$y' \in \Delta^\beta_1: \rho_{X_\beta}(x', z') = \rho_X(x,x') + 1 \le \rho_X(x',x) + 2 = \rho_X(x',y') + \rho_X(y',z')$$
		Итак, полученное пространство действительно будет метрическим. Осталось заметить, что если вложить $X$ в $X^\beta$, то $X^\beta$ будет лежать в замкнутой окрестности $X$ радиуса 1, что означает конечность расстояния между ними.
 	 \end{proof}
 	 
 	 \begin{remark}
 	 	Поскольку все облака являются собственными классами, между любыми двумя облаками существует биекция.
 	 \end{remark}
 	 
 	 \Def{Пусть $\mathcal{R}\big([X],[Y]\big)$ --- класс всех соответствий между облаками $[X]$ и $[Y]$. Определим \emph{искажение} соответствия $\dis R$ аналогично определению 1.2. \emph{Расстоянием Громова--Хаусдорфа} между облаками будем называть величину $d_{GH}\big([X],[Y]\big) = \frac{1}{2}\inf\Big(\dis R : R\in \mathcal{R}\big([X],[Y]\big)\Big)$.}
 	 
 	 
	
    %!TEX root = ../kursovaya.tex

\section{Стационарные группы и теорема о пересекающихся стационарах.}
В предыдущей части мы ввели понятие расстояния между облаками. Для дальнейшего изучения нам понадобятся некоторые вспомогательные определения. Для любого метрического пространства $X$ определена операция умножения его на положительное вещественное число $\lambda\colon X\mapsto \lambda X$, а именно $(X, \rho) \mapsto (X, \lambda \rho)$, т.е. расстояние между любыми точками пространства изменяется в $\lambda$ раз.
\begin{remark}
	Пусть метрические пространства $X$, $Y$ лежат в одном облаке. Тогда $d_{GH}(\lambda X, \lambda Y) = \lambda d_{GH}(X,Y) < \infty$, т.е. пространства $\lambda X$, $\lambda Y$ также будут лежать в одном облаке.
\end{remark} 
\Def{Определим операцию умножения облака $[X]$ на положительное вещественное число $\lambda$ как отображение, переводящее все пространства $Y \in [X]$ в пространства $\lambda Y$. По замечанию 2.1 все полученные пространства будут лежать в облаке $[\lambda X]$.}
Особый интерес представляет случай, когда такое отображение оказывается тождественным, в связи с чем вводится следующее определение.
\Def{\emph{Стационарной группой} $\St\bigl([X]\bigr)$ облака $[X]$ называется подмножество $\mathbb{R}_+$ такое, что для всех $\lambda \in \St\bigl([X]\bigr)$, $[X] = [\lambda X]$. Полученное подмножество действительно будет подгруппой в $\mathbb{R}_+$ (\cite{TuzhBog2}).}

Приведем несколько примеров облаков и их стационарных групп.

\begin{itemize}
		\item Пусть $\Delta_1$ --- одноточечное метрическое пространство. Тогда\\ $\St\bigl([\Delta_1]\bigr) = \mathbb{R}_+$.
		\item $\St\bigl([\mathbb{R}]\bigr) = \mathbb{R}_+ $.
\end{itemize}


Следующие теоремы значительно упрощают задачу по поиску расстояний между конкретными облаками.
\begin{theorem}
	Для любых облаков $[X], [Y]$ и $\lambda \in \mathbb{R}_+$ $$d_{GH}([\lambda X], [\lambda Y]) = \lambda d_{GH}([X], [Y]).$$
\end{theorem}
\begin{proof}
Пусть $R$ --- соответствие между $[X]$ и $[Y]$, а $R_\lambda$ -- соответствие между $[\lambda X]$ и $[\lambda Y]$ такие, что $(X, Y)\in R$ $\Leftrightarrow$ $(\lambda X, \lambda Y) \in R_\lambda$. Тогда\\ $$\dis{R_\lambda} = \sup\Bigl(\big||\lambda X_1 \lambda X_2| - |\lambda Y_1 \lambda Y_2|\big| : (\lambda X_1, \lambda Y_1), (\lambda X_2, \lambda Y_2) \in R_\lambda \Bigr) =$$ $$=  \lambda \sup\Bigl(\big||X_1 X_2| - |Y_1 Y_2|\big| : (X_1, Y_1), (X_2, Y_2)\in R\Bigr) = \lambda \dis{R}.$$ 
\end{proof}

\begin{corollary*}
	Если $\St([X])$, $\St([Y])$ имеют нетривиальное пересечение, т.е. $\St([X])\cap \St([Y]) \neq \{1\}$, то $d_{GH}([X], [Y]) = 0$ или $\infty$.
\end{corollary*}
\begin{proof}
	         Пусть $\lambda \in \St([X])\cap \St([Y]), \lambda \neq 1$. Тогда $[\lambda X] = [X]$, $[\lambda Y] = [Y]$ и $d_{GH}([X], [Y]) = d_{GH}([\lambda X], [\lambda Y]) =$ $ \lambda d_{GH}([X], [Y])$. Следовательно, $d_{GH}([X], [Y])$ может быть равно либо $0$, либо $\infty$.
\end{proof}

\Def{(\cite{TuzhBog2}) Если стационарная группа некоторого облака $[X]$ нетривиальна, то у него существует единственный \emph{центр} $Z\bigl([X]\bigr)$ -- это такое метрическое пространство $Y \in [X]$, что $d_{GH}(Y,\lambda Y) = 0$ для любых $\lambda \in \St \bigl([X]\bigr)$.}
    %!TEX root = ../kursovaya.tex

\section{Случай $[\Delta_1]$ и $[\mathbb{R}]$}

Вопрос об установлении точного расстояния Громова--Хаусдорфа между метрическими пространствами часто сопровождается вопросом об изометрии между этими пространствами. Известно, что изометричные пространства лежат на нулевом расстоянии друг от друга. Обратное вообще говоря не верно.

Мы рассмотрим вопрос изометрии конкретных облаков, а именно $[\Delta_1]$ и $[\mathbb{R}]$, так как эти пространства обладают рядом свойств, позволяющих значительно упростить задачу по поиску изометрии между ними.

\begin{remark}
	Центром облака $[\Delta_1]$ является одноточечный симплекс $\Delta_1$. Пусть $d_{GH}(\Delta_1, X) < l$, $d_{GH}(\Delta_1, Y) < l$, равносильно $\diam X < 2l$, $\diam Y < 2l$. Тогда $d_{GH}(X,Y) < l$ (\cite{Lectures}). Это означает, что шар радиуса $l$ с центром в $\Delta_1$ имеет диаметр $l$. 
\end{remark}

Открытый шар с центром в $x$ радиуса $d$ будем обозначать $B(x,d)$.
\begin{lemma}
	Пусть $X$ --- подмножество $\mathbb{R}$ такое, что в $\mathbb{R} \backslash X$ лежит интервал длиной $2d$. Тогда пространство $X$ лежит от $\mathbb{R}$ на расстоянии не меньшем, чем $d$.
\end{lemma}

\begin{proof}
	В $\mathbb{R} \backslash X$ лежит интервал $(a-d,a+d)$. Предположим, что $d_{GH}(\mathbb{R}, X) < d$. Пусть $(\mathbb{R}', X', Y)$ --- реализация $(\mathbb{R}, X)$ такая, что $d_H(\mathbb{R}', X') = d' < d$. Обозначим $U_1 := \cup_{x \in X', x \le a-d}B(x, d'+\frac{d-d'}{2})$, $U_2 := \cup_{x \in X', x \ge a+d}B(x, d'+\frac{d-d'}{2})$. Получаем, что $U_1, U_2$ --- два открытых непересекающихся множества, но также $\mathbb{R}' \in U_1\cup U_2$, что противоречит связности прямой.
\end{proof}

\begin{theorem}
	Образ $\Delta_1$ при изометрии не может равняться $\mathbb{R}$.
\end{theorem} 

\begin{proof}
	Для доказательства леммы достаточно предъявить два \\метрических пространства, лежащих друг от друга на расстоянии большем, чем максимум их расстояний до $\mathbb{R}$. Рассмотрим $\mathbb{R}$ как подмножество $\mathbb{R}^2$ и добавим к нему точку $(0,1)$, расстояние до которой будет соответствовать манхэттенской метрике в $\mathbb{R}^2$. Обозначим это пространство $\widetilde{\mathbb{R}}$, тогда $d_{GH}(\mathbb{R}, \widetilde{\mathbb{R}}) \le \frac{1}{2}$. Также $d_{GH}(\mathbb{R},\mathbb{Z}) \le \frac{1}{2}$. Пусть $R$ --- соответствие между $\mathbb{Z}$ и $\widetilde{\mathbb{R}}$, $\dis R < 1 + \epsilon$ и $\big(i, (0,1)\big) \in R$. Тогда, если $x \in \mathbb{R}\backslash(-\epsilon, \epsilon)$, то $(i, x) \notin R$. Обозначим за $\mathcal{N}$ множество всех $k \in \mathbb{Z}$ таких, что для всех $x \in \mathbb{R}\backslash(-\epsilon, \epsilon)$, $(k,x) \notin R$. $\mathcal{N}$ не пусто и не равно $\mathbb{Z}$, следовательно, по лемме 3.1, расстояние от $\mathbb{Z} \backslash \mathcal{N}$ до $\mathbb{R}$ будет не меньше 1. Рассмотрим подмножество соответствия $R$, $R' := R \backslash \{(k,x): x\in (-\epsilon, \epsilon), k \in \mathbb{Z}\} \cup \{\big(i, (0,1)\big)\}$. Так как все точки из $\mathbb{R}\backslash(-\epsilon, \epsilon)$ лежат в $R$ только в паре с точками из $ \mathbb{Z} \backslash \mathcal{N}$ и наоборот, множество $R'$ будет соответствием между $\mathbb{R}\backslash(-\epsilon, \epsilon)$ и $ \mathbb{Z} \backslash \mathcal{N}$  Тогда, $\dis R \ge \dis R' \ge 2d_{GH}(\mathbb{R}\backslash(-\epsilon, \epsilon), \mathbb{Z} \backslash \mathcal{N})$ По неравенству треугольника $2d_{GH}(\mathbb{R}\backslash(-\epsilon, \epsilon), \mathbb{Z} \backslash \mathcal{N}) \ge 2|d_{GH}(\mathbb{R}, \mathbb{Z} \backslash \mathcal{N}) - d_{GH}(\mathbb{R}\backslash(-\epsilon, \epsilon), \mathbb{R})| \ge 2 - 2\epsilon$, что при малых $\epsilon$ больше, чем $1 + \epsilon$. Получаем, что $d_{GH}(\widetilde{\mathbb{R}}, \mathbb{Z}) > \frac{1}{2}$.    
\end{proof}

\begin{corollary}
	Облака $[\Delta_1]$ и $[\mathbb{R}]$ не изометричны.
\end{corollary}

\begin{proof}
	Предположим, что существует изометрия $F\colon$$[\Delta_1] \mapsto [\mathbb{R}]$. По предыдущей теореме получили, что $F(\Delta_1) = X \neq \mathbb{R}$. Рассмотрим пространство $\lambda X$, которое будет являться образом пространства $\widetilde{X}$ из $[\Delta_1]$, отличного от $\Delta_1$. Через $\widetilde{X}$ в облаке $[\Delta_1]$ проходит геодезическая, а значит через $\lambda X$ проходит образ этой геодезической. Это означает, что найдутся пространства $Y_1, Y_2$ такие, что $d_{GH}(\lambda X, Y_1) = d_{GH}(\lambda X, Y_2) = d$ и $d_{GH}(Y_1, Y_2) = 2d$. Поделим эти равенства на $\lambda: $ $d_{GH}(X, \frac{1}{\lambda} Y_1) = d_{GH}(X, \frac{1}{\lambda} Y_2) = \frac{d}{\lambda}$ и $d_{GH}(\frac{1}{\lambda} Y_1, \frac{1}{\lambda} Y_2) = \frac{2d}{\lambda}$. Теперь, если мы рассмотрим прообразы пространств $X, \frac{1}{\lambda} Y_1, \frac{1}{\lambda} Y_2$ при изометрии $F$, то получим противоречие с замечанием 3.1.
\end{proof}




    %!TEX root = ../kursovaya.tex

\section{Подсчет расстояния между облаками $[\Delta_1]$ и $[\mathbb{R}]$}
\begin{theorem}
	Расстояния между облаками $[\Delta_1]$ и $[\mathbb{R}]$ равно бесконечности.
\end{theorem} 

\begin{proof}
	Так как у облаков $[\Delta_1]$ и $[\mathbb{R}]$ совпадают стационарные группы, по следствию 1 расстояние между ними может быть равно либо $0$, либо $\infty$. Покажем, что оно не равно $0$. Для этого необходимо установить, что между ними не может существовать соответствия со сколь угодно малым искажением. Итак, пусть $R$ --- соответствие между $[\Delta_1]$ и $[\mathbb{R}]$, $\dis R = \epsilon$.
	
	$\{X_\alpha\}$ --- $\epsilon$-сеть в $B(\Delta_1,1)$ такая, что $|X_{\alpha_1}, X_{\alpha_2}| > \epsilon$ для любых $\alpha_1, \alpha_2$. Значит за $P(X_\alpha)$ множество всех $Y\in \mathbb{R}$ таких, что $(X_\alpha, Y)\in R$.
	
	Если $Y_1, Y_2 \in P(X_\alpha)$, то $|Y_1, Y_2| \le \epsilon$. Если $Y \in P(X_{\alpha_1}) \cap P(X_{\alpha_2})$, то $|X_{\alpha_1}, X_{\alpha_2}| \le \epsilon$, что противоречит выбору $X_\alpha$. Следовательно для любых $\alpha_1, \alpha_2$, $P(X_{\alpha_1})$ не пересекается с $P(X_{\alpha_2})$, и в каждом $P(X_\alpha)$ можно выбрать уникальный $Y_\alpha$. Далее под $P(X_\alpha)$ будем подразумевать именно этот $Y_\alpha$.
	\begin{lemma}
		$P(\Delta_1)$ лежит от $\mathbb{R}$ на расстоянии, не большем $2\epsilon$.
	\end{lemma}
	\begin{proof}
		Если $P(\Delta_1)$ не совпадает с $\mathbb{R}$, то через $P(\Delta_1)$ проходит геодезическая, и существуют пространства $Y_1, Y_2 \in [\mathbb{R}]$ такие, что $|Y_1,P(\Delta_1)| = |Y_2,P(\Delta_1)| = \rho$, $|Y_1,Y_2| = 2\rho$, причем $\rho$ можно взять сколь угодно б. Существуют $X_1, X_2\in [\Delta_1]$, лежащие в соответствии $R$ в паре с $Y_1, Y_2$ соответственно. Тогда для них верны следующие неравенства 
		$$\rho - \epsilon \le |X_1, \Delta_1| \le \rho + \epsilon$$
		$$\rho - \epsilon \le |X_2, \Delta_1| \le \rho + \epsilon$$
		Из этих неравенств получаем, что $\rho - \epsilon \le |X_1, X_2| \le \rho + \epsilon$. Вычитая последнее неравенство из расстояния между $|Y_1, Y_2|$ получаем $$\rho - \epsilon \le |Y_1,Y_2| - |X_1,X_2|\le \rho + \epsilon$$
		При этом из того, что искажение соответствия $R$ равно $\epsilon$ следует
		$$\big||X_1,X_2| - |Y_1, Y_2|\big| \le \epsilon$$ 
		Если $\rho > \epsilon$, то из последних двух неравенств следует, что $\rho < 2\epsilon$. Так как $\rho$ можно выбрать сколь угодно близко к $|\mathbb{R}, P(\Delta_1)|$, получаем, что $|\mathbb{R}, P(\Delta_1)| \le 2\epsilon$.
	\end{proof}
	Продолжим доказательство теоремы.
	
	Если взять $B(\mathbb{R}, r)$, то начиная с некоторого $\epsilon$ $\{P(X_\alpha)\}$ будет $2\epsilon$- сетью в $B(\mathbb{R}, r)$. Возьмем пространства $Y_1, Y_2\in B(\mathbb{R}, r)$ такие, что $|Y_1,\mathbb{R}| = |Y_2,\mathbb{R}| = \rho$, $|Y_1,Y_2| = 2\rho$. Так как $\{P(X_\alpha)\}$, существуют такие $X_{\alpha_1}, X_{\alpha_2}\in [\Delta_1]$ для которых $|Y_1,P(X_{\alpha_1})| \le 2\epsilon, |Y_2,P(X_{\alpha_2})| \le 2\epsilon$. Из этих неравенств получаем следующее
	
	$$|P(X_{\alpha_1}), P(X_{\alpha_2})| \ge 2\rho - 4\epsilon$$
	$$|P(X_{\alpha_i}), P(\Delta_1)| \le \rho + 4\epsilon, i = 1, 2$$
	
	Так, как $P$ -- биекция, можно взять обратное отображение $P^{-1}$, при этом расстояния не могут измениться больше, чем на $\epsilon$. $\rho$ не зависит от $\epsilon$, и при $\epsilon \ll \rho < 1$ получим, что система неравенств выше не выполняется. Получили противоречие, следовательно соответствие со сколь угодно малым искажением не может существовать. Значит расстояние между облаками $[\Delta_1]$ и $[\mathbb{R}]$ может быть равно только бесконечности, что завершает доказательство.
\end{proof}
}


% Информация о годе выполнения работы
\def\Year{%
    % 2006%
    \the\year%     % Текущий год
}

% Укажите тип работы
% Например:
%     Выпускная квалификационная работа,
%     Магистерская диссертация,
%     Курсовая работа, реферат и т.п.
\def\WorkType{%
    % Выпускная квалификационная работа%
    % Магистерская диссертация%
     Курсовая работа%
    % Реферат%
    %Дипломная работа%
}

% Название работы
%%%%%%%%%%% ВНИМАНИЕ! %%%%%%%%%%%%%%%%
% В МГУ ОНО ДОЛЖНО В ТОЧНОСТИ
% СООТВЕТСТВОВАТЬ ВЫПИСКЕ ИЗ ПРИКАЗА
% УТОЧНИТЕ НАЗВАНИЕ В УЧЕБНОЙ ЧАСТИ
\def\Title{%
    Исследование расстояния Громова-Хаусдорфа между облаками%
}


% Имя автора работы
\def\Author{%
    Нестеров Борис Аркадьевич%
}

% Информация о научном руководителе
%% Фамилия Имя Отчество%
\def\SciAdvisor{%
    Тужилин Алексей Августинович%
}
%% В формате: И.~О.~Фамилия%
\def\SciAdvisorShort{%
    А.~А.~Тужилин%
}
%% должность научного руководителя
\def\Position{%
     профессор%
    %доцент%
    % старший преподаватель%
    % преподаватель%
    % ассистент%
    % ведущий научный сотрудник%
    % старший научный сотрудник%
    % научный сотрудник%
    % младший научный сотрудник%
}
%% учёная степень научного руководителя
\def\AcademicDegree{%
     д.ф.-м.н.%
    % д.т.н.%
    %к.ф.-м.н.%
    % к.т.н.%
    % без степени%
}

% Информация об организации, в которой выполнена работа
%% Город
\def\Place{%
    Москва%
}
%% Университет
\def\Univer{%
    Московский государственный университет имени М.~В.~Ломоносова%
}
%% Факультет
\def\Faculty{%
    Механико-математический факультет%
}
%% Кафедра    
\def\Department{%
    Кафедра дифференциальной геометрии и приложений%
}     

%%%% Переключите статус документа для отладки
%%%% В режиме draft документ собирается очень быстро
%%%% и выводится полезная информация о том
%%%% какие строки вылезают за границы документа, что удобно для борьбы с ними
\def\Status{%
     %draft%
    final%
}

%%%% Включает и выключает подпись <<С текстом работы ознакомлен>>
\def\EnableSign{%
    % true%
}
