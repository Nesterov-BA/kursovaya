  %!TEX root = ../kursovaya.tex

\section{Основные определения и предварительные результаты} Пусть $X$ и $Y$ ---
метрические пространства. Тогда между ними можно задать расстояние, называемое
расстоянием Громова--Хаусдорфа. Введем два его эквивалентных (\cite{Lectures})
определения.  \Def{Пусть $X$, $Y$ --- метрические пространства с метриками
$\rho_X$ и $\rho_Y$. \emph{Соответствием} $R$ между этими пространствами
называется подмножество декартового произведения $X\times Y$ такое, что
проекторы $\pi_X \colon (x,y) \mapsto x$, $\pi_Y\colon (x,y) \mapsto y$ являются
сюрьективными. Множество всех соответствий между $X$ и $Y$ обозначается
$\mathcal{R}(X,Y).$} \Def{ Пусть $R$ --- соответствие между $X$ и $Y$.
\emph{Искажением} соответствия $R$, $\dis{R}$ является
величина $$ \dis{R} = \sup{\Bigl\{ \big| |xx'| - |yy'| \big| : (x, y), (x', y') \in R\Bigr\}}.$$
Тогда \emph{расстояние Громова--Хаусдорфа} $d_{GH}(X,Y)$ можно определить
следующим
образом $$ d_{GH}(X,Y) = \frac{1}{2}\inf \bigl\{\dis{R} : R \in \mathcal{R}(X,Y)\bigr\}.$$}
	
	\Def{\emph{Реализацией} пары метрических пространств $(X,Y)$ назовем тройку
метрических пространств $(X',Y',Z)$ таких, что $X' \subset Z$, $Y' \subset Z$,
$X'$ изометрично $X$, $Y'$ изометрично $Y$. \emph{Расстоянием Громова-Хаусдорфа}
$d_{GH}(X,Y)$ между метрическими пространствами $X, Y$ является точная нижняя
грань чисел $r$ таких, что существует реализация $(X',Y',Z)$ и \\$d_H(X', Y')
\le r$, где $d_H$ --- расстояние Хаусдорфа.}

	Рассмотрим собственный класс всех метрических пространств и отождествим в нем между собой все метрические пространства, находящиеся на нулевом расстоянии друг от друга. Обозначим получившийся класс $\mathcal{GH}_0$. На нем расстояние Громова -- Хаусдорфа будет являться обобщенной метрикой.
	
	\Def{(\cite{TuzhBog1}) В классе $GH$ рассмотрим следующее отношение:
$X \thicksim Y \Leftrightarrow d_{GH}(X, Y) < \infty$. Нетрудно убедиться, что
оно будет отношением эквивалентности. Классы этой эквивалентности называются
\emph{облаками}. Облако, в котором лежит метрическое пространство $X$ будем
обозначать $[X]$.}
	 


 	 Для любого метрического пространства $X$ определена операция умножения его
на положительное вещественное число $\lambda\colon X\mapsto \lambda X$, а именно
$(X, \rho) \mapsto (X, \lambda \rho)$, расстояние между любыми точками
пространства изменяется в $\lambda$ раз.
 	 \begin{remark} Пусть метрические пространства $X$, $Y$ лежат в одном
облаке. Тогда $d_{GH}(\lambda X, \lambda Y) = \lambda d_{GH}(X,Y) < \infty$,
т.е. пространства $\lambda X$, $\lambda Y$ также будут лежать в одном облаке.
 	 \end{remark}  \Def{Определим операцию умножения облака $[X]$ на
положительное вещественное число $\lambda$ как отображение, переводящее все
пространства $Y \in [X]$ в пространства $\lambda Y$. По замечанию 1.2 все
полученные пространства будут лежать в облаке $[\lambda X]$.}
При таком отображении облако может как измениться, так и перейти в себя. Для последнего случая вводится специальное определение.
\Def{\emph{Стационарной группой}
$\St\bigl([X]\bigr)$ облака $[X]$ называется подмножество $\mathbb{R}_+$ такое,
что для всех $\lambda \in \St\bigl([X]\bigr)$, $[X] = [\lambda X]$. Полученное
подмножество действительно будет подгруппой в $\mathbb{R}_+$ (\cite{TuzhBog2}).}

Приведем несколько примеров облаков и их стационарных групп.
 	 
 	 \begin{itemize}
 	 	\item Пусть $\Delta_1$ --- одноточечное метрическое пространство.
Тогда\\ $\St\bigl([\Delta_1]\bigr) = \mathbb{R}_+$.
 	 	\item $\St\bigl([\mathbb{R}]\bigr) = \mathbb{R}_+ $.
 	 \end{itemize} \Def{(\cite{TuzhBog2}) Если стационарная группа некоторого
облака $[X]$ нетривиальна, то у него существует единственный \emph{центр}
$Z\bigl([X]\bigr)$ -- это такое метрическое пространство $Y \in [X]$, что
$d_{GH}(Y,\lambda Y) = 0$ для любых $\lambda \in \St \bigl([X]\bigr)$.}  Далее
за $R(X)$ будем обозначать образ пространства $X$ при
соответствии $R$, то есть
$R(X) = \{Y \in [\mathbb{R}] | (X, Y) \in R\}$. Под образом пространства будет
иметься ввиду именно образ при соответствии.  Аналогично определим $R^{-1}(Y)$.

 Для любых пространств $Y_1, Y_2 \in R(X)$ выполнено неравенство
$|Y_1, Y_2| = \big| |Y_1,Y_2| - |X,X|\big| \le \dis R$, откуда следует, что
диаметр  $R(X)$ не превосходит $\dis R$. Из этого, в частности, следует, что
пространства, лежащие на расстоянии большем, чем искажение соответствия, не
могут принадлежать образу или прообразу одного пространства.\\ По определению
искажения, если расстояние между пространствами $X_1, X_2$ равно $\rho$, то
расстояние между пространствами в их образах будет отличаться от $\rho$ не более
чем на $\dis R$. Это же верно и для пространств в прообразах.
 \begin{remark} В облаке $[\Delta_1]$ для любого пространства $X$ выполняется:
	$$|\lambda X, \mu X| = |\lambda - \mu||X,\Delta_1|.$$
 \end{remark}
 \begin{remark}[Ультраметрическое неравенство] В облаке $[\Delta_{1}]$ для всех пространств $X_{1}, X_{2}$
выполняется неравенство:
   $$|X_{1},X_{2}| \le \max\{|X_{1}, \Delta_{1}|,|X_{2},\Delta_{1}|\}$$
 \end{remark}
