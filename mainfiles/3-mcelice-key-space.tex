%!TEX root = ../kursovaya.tex

\section{Стационарные группы и теорема о пересекающихся стационарах.}
В предыдущей части мы ввели понятие расстояния между облаками. Для дальнейшего изучения нам понадобятся некоторые вспомогательные определения. Для любого метрического пространства $X$ определена операция умножения его на положительное вещественное число $\lambda\colon X\mapsto \lambda X$, а именно $(X, \rho) \mapsto (X, \lambda \rho)$, т.е. расстояние между любыми точками пространства изменяется в $\lambda$ раз.
\begin{remark}
	Пусть метрические пространства $X$, $Y$ лежат в одном облаке. Тогда $d_{GH}(\lambda X, \lambda Y) = \lambda d_{GH}(X,Y) < \infty$, т.е. пространства $\lambda X$, $\lambda Y$ также будут лежать в одном облаке.
\end{remark} 
\Def{Определим операцию умножения облака $[X]$ на положительное вещественное число $\lambda$ как отображение, переводящее все пространства $Y \in [X]$ в пространства $\lambda Y$. По замечанию 2.1 все полученные пространства будут лежать в облаке $[\lambda X]$.}
Особый интерес представляет случай, когда такое отображение оказывается тождественным, в связи с чем вводится следующее определение.
\Def{\emph{Стационарной группой} $\St\bigl([X]\bigr)$ облака $[X]$ называется подмножество $\mathbb{R}_+$ такое, что для всех $\lambda \in \St\bigl([X]\bigr)$, $[X] = [\lambda X]$. Полученное подмножество действительно будет подгруппой в $\mathbb{R}_+$ (\cite{TuzhBog2}).}

Приведем несколько примеров облаков и их стационарных групп.

\begin{itemize}
		\item Пусть $\Delta_1$ --- одноточечное метрическое пространство. Тогда\\ $\St\bigl([\Delta_1]\bigr) = \mathbb{R}_+$.
		\item $\St\bigl([\mathbb{R}]\bigr) = \mathbb{R}_+ $.
\end{itemize}


Следующие теоремы значительно упрощают задачу по поиску расстояний между конкретными облаками.
\begin{theorem}
	Для любых облаков $[X], [Y]$ и $\lambda \in \mathbb{R}_+$ $$d_{GH}([\lambda X], [\lambda Y]) = \lambda d_{GH}([X], [Y]).$$
\end{theorem}
\begin{proof}
Пусть $R$ --- соответствие между $[X]$ и $[Y]$, а $R_\lambda$ -- соответствие между $[\lambda X]$ и $[\lambda Y]$ такие, что $(X, Y)\in R$ $\Leftrightarrow$ $(\lambda X, \lambda Y) \in R_\lambda$. Тогда\\ $$\dis{R_\lambda} = \sup\Bigl(\big||\lambda X_1 \lambda X_2| - |\lambda Y_1 \lambda Y_2|\big| : (\lambda X_1, \lambda Y_1), (\lambda X_2, \lambda Y_2) \in R_\lambda \Bigr) =$$ $$=  \lambda \sup\Bigl(\big||X_1 X_2| - |Y_1 Y_2|\big| : (X_1, Y_1), (X_2, Y_2)\in R\Bigr) = \lambda \dis{R}.$$ 
\end{proof}

\begin{corollary*}
	Если $\St([X])$, $\St([Y])$ имеют нетривиальное пересечение, т.е. $\St([X])\cap \St([Y]) \neq \{1\}$, то $d_{GH}([X], [Y]) = 0$ или $\infty$.
\end{corollary*}
\begin{proof}
	         Пусть $\lambda \in \St([X])\cap \St([Y]), \lambda \neq 1$. Тогда $[\lambda X] = [X]$, $[\lambda Y] = [Y]$ и $d_{GH}([X], [Y]) = d_{GH}([\lambda X], [\lambda Y]) =$ $ \lambda d_{GH}([X], [Y])$. Следовательно, $d_{GH}([X], [Y])$ может быть равно либо $0$, либо $\infty$.
\end{proof}

\Def{(\cite{TuzhBog2}) Если стационарная группа некоторого облака $[X]$ нетривиальна, то у него существует единственный \emph{центр} $Z\bigl([X]\bigr)$ -- это такое метрическое пространство $Y \in [X]$, что $d_{GH}(Y,\lambda Y) = 0$ для любых $\lambda \in \St \bigl([X]\bigr)$.}