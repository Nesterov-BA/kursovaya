%!TEX root = ../kursovaya.tex

\section{Подсчет расстояния между облаками $[\Delta_1]$ и $[\mathbb{R}]$}

\begin{theorem} Пусть у облака $[Z]$ нетривиальнвя стационарная группа, и $Z$
является его центром. Также, пусть в этом облаке есть пространства $Y_{1}, Y_{2}$
такие, что $\max\big\{ |Y_{1},Z|, |Y_{2}, Z| \big\} = r>0$, а
$|Y_{1}, Y_{2}|>r$. Тогда, расстояние между облаками $[\Delta_1]$ и $[Z]$
равно бесконечности.
\end{theorem} 
\begin{proof} У облаков $[\Delta_1]$ и $[Z]$ стационарные группы имеют
нетривиальное пересечение, и, по следствию 1, расстояние между ними может быть
равно либо $0$, либо $\infty$. \\ Для доказательства утверждения теоремы
достаточно будет показать, что расстояние между ними не равно $0$.  Для этого
необходимо установить, что между ними не может существовать соответствия со
сколь угодно малым искажением. Итак, пусть $R$ --- соответствие между
$[\Delta_1]$ и $[Z]$, $\dis R = \epsilon < \infty$.
Зафиксируем $Y$ из $R(\Delta_1)$. По теореме 2.1 расстояние между $Y$ и $Z$ не
больше $2\epsilon$.\\ По условию теоремы выполнено неравенство:
 $$\max\big\{ |Y_{1},Z|, |Y_{2}, Z| \big\} = r < |Y_{1}, Y_{2}|$$
Неравенство означает, что существует $c > 0$ такое, что
$|Y_{1},Y_{2}| = (1 + c)r.$ \\ Вместе с $Y_{1}$ и
$ Y_{2}$ рассмотрим их прообразы $X_1 \in R^{-1}(Y_{1})$,
$ X_2 \in R^{-1}(Y_{2})$.  \\ Получаем следующую цепочку
неравенств:
	$$|X_1, \Delta_1| \le |Y_{1}, Y| + \epsilon \le |Y_{1}, \mathbb{R}| + |\mathbb{R}, Y| +\epsilon \le r + 2\epsilon + \epsilon = r + 3\epsilon.$$
	Аналогичное неравенство имеет место для $X_2$, при этом
	$$|X_1, X_2|  \ge |Y_{1}, Y_{2}| - \epsilon = (1+c)r - \epsilon.$$
	По замечанию 1.4:
	$$|X_1, X_2| \le \max\big\{ |X_1, \Delta_1|, |X_2, \Delta_1| \big\},$$
	$$\Updownarrow$$
	$$(1+c)r - \epsilon\le r + 3\epsilon,$$
	$$\Updownarrow$$
	$$\epsilon \ge \frac{cr}{4}.$$
	Мы получаем оценку снизу для $\epsilon = \dis R$. Это означает, что
искажение не может быть произвольно малым, и следовательно расстояние между
пространствами не может быть равно 0. Значит, оно равно бесконечности.
	
\end{proof}
\begin{corollary}
	В облаке $[\mathbb{R}]$ в качестве пространств $Y_{1}, Y_{2}$ можно взять
$\mathbb{Z}, \widetilde{\mathbb{R}}$. Для них, по теореме 3.1 будет выполнено
неравенство из условия теоремы 4.1 с $r = \frac 1 2$. Стационарная группа облака
$[\mathbb{R}]$ равна $\mathbb{R}^{+}$, то есть нетривиальна. Получаем, что расстояние между облаками
$[\Delta_{1}]$ и $[\mathbb{R}]$ равно $\infty$.
\end{corollary}
