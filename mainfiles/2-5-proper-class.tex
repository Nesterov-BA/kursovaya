\section{Мощность облаков}
Метрические пространства по своему определению являются множествами.
Соответственно для переноса конструкции расстояния Громова-Хаусдорфа на облака,
необходимо либо установить, что они --- множества, либо \\ соответствующим
образом изменить определение расстояния.
    \begin{theorem} Все облака представляют собой собственные классы.
	\end{theorem}
	\begin{proof} Для доказательства теоремы достаточно показать, что в любом
облаке лежат пространства сколь угодно большой мощности.  Пусть $X$ --
метрическое пространство мощности $\alpha$. Расширим это пространство до
пространства большей мощности. Обозначим $\Delta^\beta_1$ --- симплекс мощности
$\beta$, где $\beta > \alpha$. Обозначим $X^\beta = X \cup \Delta^\beta_1$.
Зафиксируем произвольную точку $x$ пространства $X$ и положим расстояние от нее
до любой точки симплекса равным $1$. Для точек $x' \in X$,
$y \in \Delta^\beta_1$ определим
$$\rho_{X_\beta}(y,x') = \rho_{X_\beta}(x',y) := \rho_X(x',x) + 1.$$
Расстояния между другими парами точек оставим без изменений.
Симметричность и неотрцательность расстояния $\rho_{X_{\beta}}$ очевидны.
Для того, чтобы
полученное расстояние являлось метрикой достаточно проверить выполнение
неравенства треугольника
$\rho_{X_\beta}(x',z') \le \rho_{X_\beta}(x',y') +\rho_{X_\beta}(y',z')$
только в том случае, если точки $x', y', z'$ не лежат одновременно в
$\Delta^\beta_1$ или в $X$. Случаи $x', z' \in \Delta^\beta_1$ и $ x', z' \in X$
очевидны. Разберем подробнее случаи, когда $x' \in X, z' \in \Delta^\beta_1$:
		$$ y' \in X: \rho_{X_\beta}(x', z') = \rho_X(x,x') + 1 \le \rho_X(x,y') + \rho_X(y',x') + 1 = \rho_X(x',y') + \rho_X(y',z')$$
		$$y' \in \Delta^\beta_1: \rho_{X_\beta}(x', z') = \rho_X(x,x') + 1 \le \rho_X(x',x) + 2 = \rho_X(x',y') + \rho_X(y',z')$$
		Итак, полученное пространство действительно будет метрическим. Осталось
заметить, что если вложить $X$ в $X^\beta$, то $X^\beta$ будет лежать в
замкнутой окрестности $X$ радиуса 1, что означает конечность расстояния между
ними.
 	 \end{proof}

 	 \begin{remark} Поскольку все облака являются собственными классами, между
любыми двумя облаками существует биекция. Это означает, в частности, что класс соответствий между любыми двумя облаками не пуст.
 	 \end{remark}

 	 \Def{Пусть $\mathcal{R}\big([X],[Y]\big)$ --- класс всех соответствий между
облаками $[X]$ и $[Y]$. Определим \emph{искажение} соответствия $\dis R$
аналогично определению 1.2. \emph{Расстоянием Громова--Хаусдорфа} между облаками
будем называть величину
$d_{GH}\big([X],[Y]\big) = \frac{1}{2}\inf\bigl\{\dis R : R\in \mathcal{R}\big([X],[Y]\big)\bigr\}$.}
