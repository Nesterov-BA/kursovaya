%!TEX root = ../kursovaya.tex

\section{Теорема об образе центра}
\begin{theorem}
Пусть $M$ -- центр облака $[M]$, имеющего нетривиальную
\\стационарную группу. $R$ -- соответствие между $[\Delta_{1}]$ и $[M]$ с конечным
искажением $\epsilon$. Тогда образ пространства $\Delta_{1}$ лежит от $M$ на
расстоянии не большем $2\epsilon$.
\end{theorem}
\begin{proof}
Нетривиальность стационарной группы $[M]$ означает, что найдется
число $l > 1$ такое, что $\{l^{j}|j\in \mathbb{Z}\}$ является подгруппой в
$\St{[M]}$.\\
Зафиксируем $Y$ из образа $\Delta_{1}$.
Предположим, что $|M, Y| = d > \epsilon$.  Обозначим
$|Y, kY| = \rho$, $k \ge 2$, $k = l^{j_{1}}$. По неравенству треугольника $\rho + d \ge kd$,
откуда $\rho \ge (k-1)d > (k-1)\epsilon$. Тогда $kY$ лежит в образе
$X \ne \Delta_1$. При этом,
$\rho - \epsilon \le |X, \Delta_1| \le \rho + \epsilon$. \\
Возьмем произвольные $\alpha > 0$ и $\beta \in (0,1)$. Для пространств $(1+\alpha)X, (1-\beta)X$ будут выполняться неравенства:
	$$|X, (1+\alpha)X| = \alpha |X, \Delta_1| \le \alpha\rho + \alpha\epsilon,$$
	$$|X, (1-\beta)X| = \beta|X, \Delta_1| \le \beta\rho + \beta\epsilon,$$
	$$|(1+\alpha) X, (1-\beta)X| = (\alpha + \beta)|X, \Delta_1| \ge (\alpha+\beta)\rho - (\alpha+\beta)\epsilon.$$
 Существуют
$Y_\alpha, Y_\beta \in [M]$ такие, что
$kY_\alpha \in R\big((1+\alpha)X\big)$, $kY_\beta \in R\big((1-\beta)X\big)$, и
для них выполняются следующие неравенства:
	$$|kY, kY_\alpha| \le |X, (1+\alpha)X| + \epsilon \le \alpha\rho + (\alpha+1)\epsilon,$$
	$$|kY, kY_\beta| \le |X, (1-\beta)X| + \epsilon \le \beta\rho + (\beta+1)\epsilon,$$
	$$|kY_\alpha, kY_\beta| \ge |(1+\alpha)X, (1-\beta)X| - \epsilon \ge  (\alpha+\beta)\rho - (\alpha+\beta+1)\epsilon.$$
	Поделим эти неравенства на $k$:
	$$|Y, Y_{\alpha}| \le \frac{\alpha}{k}\rho + \frac{\alpha+1}{k}\epsilon,$$
	$$|Y, Y_{\beta}| \le \frac{\beta}{k}\rho + \frac{\beta+1}{k}\epsilon,$$
	$$|Y_\alpha, Y_{\beta}| \ge \frac{\alpha+\beta}{k}\rho - \frac{\alpha+\beta+1}{k}\epsilon.$$
	и возьмем прообразы пространств $Y, Y_{\alpha}, Y_{\beta}$:
	$$|\Delta_1, X_{\alpha}| \le \frac{\alpha}{k}\rho + \big(\frac{\alpha+1}{k} + 1\big)\epsilon,$$
	$$|\Delta, X_{\beta}| \le \frac{\beta}{k}\rho + \big(\frac{\beta+1}{k}+1\big)\epsilon,$$
	$$|X_\alpha, X_{\beta}| \ge \frac{\alpha+\beta}{k}\rho - \big(\frac{\alpha+\beta+1}{k}+1\big)\epsilon.$$
	 Считая, что $\alpha > \beta$ получаем неравенство:
	 $$\frac{\alpha+\beta}{k}\rho - \big(\frac{\alpha+\beta+1}{k}+1\big)\epsilon \le \frac{\alpha}{k}\rho + \big(\frac{\alpha+1}{k} + 1\big)\epsilon,$$
	 $$\Updownarrow$$
	 $$\rho \le \frac{k}{\beta}\bigg(\frac{2\alpha+\beta+2}{k}+2\bigg)\epsilon,$$
	 $$\Updownarrow$$
	 $$\rho \le \bigg(1+\frac{2\alpha + 2}{\beta} + 2\frac{k}{\beta}\bigg)\epsilon.$$
	Нас интересует оценка сверху для $d$:
	$$d \le \frac{\rho}{k-1} \le \bigg(\frac{1}{k-1}+\frac{2\alpha + 2}{\beta(k-1)} + 2\frac{k}{\beta(k-1)}\bigg)\epsilon $$
	Последнее слагаемое в скобках строго больше 2 при любых $k>2$, $\alpha>0$,
$\beta\in (0,1)$, а остальные слагаемые с ростом $k$ стремятся к $0$. Так как стационарная группа нетривиальна, в ней есть последовательности чисел стремящихся к 0 и к $\infty$.
Устремив $\beta$ к 1, а $k$ к бесконечности получаем оценку:
	$$|Y, M| \le 2\epsilon.$$

\end{proof}
