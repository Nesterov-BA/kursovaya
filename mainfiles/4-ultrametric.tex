%!TEX root = ../kursovaya.tex

\section{Невыполнение ультраметрического неравенства}
Для облака $[\Delta_{1}]$ справедливо ультраметрическое неравенство(Замечание
1.4). Следующая лемма показывает, что для облака $[\mathbb[R]]$
это неравенство может не выполняться. \\
Рассмотрим $\mathbb{R}$ как подмножество
$\mathbb{R}^2$ и добавим к нему точку $(0,1)$, расстояние до которой будет
соответствовать манхэттенской метрике в $\mathbb{R}^2$. Обозначим это
пространство $\widetilde{\mathbb{R}}$.
\begin{theorem}
\begin{enumerate}
\itemПространства $\mathbb{Z}$ и $\widetilde{\mathbb{R}}$ находятся от $\mathbb{R}$
на расстоянии\\ не большем $\frac 1 2$.
\item Расстояние между $\mathbb{Z}$ и $\widetilde{\mathbb{R}}$ строго
больше $\frac 1 2$.
\end{enumerate}
\end{theorem}
\begin{proof}
Вложением целых чисел в вещественную прямую получается реализация $\mathbb{Z}$, $\mathbb{R}$ с расстоянием Хаусдорфа равным $\frac 1 2$.
Если вложить $\widetilde{\mathbb{R}}$ в $\mathbb{R}^{2}$ естественным образом, а $\mathbb{R}$ вложить как подмножество равное $\{(x, \frac1 2 )|x\in \mathbb{R}\}$,
расстояние Хаусдорфа между ними также будет равно $\frac 1 2$. Таким образом, первое утверждение теоремы доказано. \\
Пусть $R$ --- соответствие
между $\mathbb{Z}$ и $\widetilde{\mathbb{R}}$, $\dis R < 1 + \epsilon$ и
$\big(i, (0,1)\big) \in R$. Тогда, если
$x \in \mathbb{R}\backslash(-\epsilon, \epsilon)$, то $(i, x) \notin R$.
Обозначим за $\mathcal{N}$ множество всех $k \in \mathbb{Z}$ таких, что для всех
$x \in \mathbb{R}\backslash(-\epsilon, \epsilon)$, $(k,x) \notin R$.
$\mathcal{N}$ не пусто и не равно $\mathbb{Z}$, следовательно, по лемме 3.1,
расстояние от $\mathbb{Z} \backslash \mathcal{N}$ до $\mathbb{R}$ будет не
меньше 1. Рассмотрим подмножество соответствия $R$,
$R' := R \backslash \{(k,x): x\in (-\epsilon, \epsilon), k \in \mathbb{Z}\} \cup \{\big(i, (0,1)\big)\}$.
Так как все точки из $\mathbb{R}\backslash(-\epsilon, \epsilon)$ лежат в $R$
только в паре с точками из $ \mathbb{Z} \backslash \mathcal{N}$ и наоборот,
множество $R'$ будет соответствием между
$\mathbb{R}\backslash(-\epsilon, \epsilon)$ и
$ \mathbb{Z} \backslash \mathcal{N}$  Тогда,
$\dis R \ge \dis R' \ge 2d_{GH}(\mathbb{R}\backslash(-\epsilon, \epsilon), \mathbb{Z} \backslash \mathcal{N})$
По неравенству треугольника
$2d_{GH}(\mathbb{R}\backslash(-\epsilon, \epsilon), \mathbb{Z} \backslash \mathcal{N}) \ge 2|d_{GH}(\mathbb{R}, \mathbb{Z} \backslash \mathcal{N}) - d_{GH}(\mathbb{R}\backslash(-\epsilon, \epsilon), \mathbb{R})| \ge 2 - 2\epsilon$,
что при малых $\epsilon$ больше, чем $1 + \epsilon$. Получаем, что
$d_{GH}(\widetilde{\mathbb{R}}, \mathbb{Z}) > \frac{1}{2}$.
\end{proof}
